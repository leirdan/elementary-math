\enquote{Determine o(s) valor(es) de $m \in \R$ tal(is) que a equação $mx^2 + (m + 1)x + (m + 1) = 0$ tenha somente uma raiz real.}
\\
\emph{Resolução.} Calculemos o discriminante desta equação, a princípio:
\begin{align*}
    \Delta = b^2 - 4ac & \implies (m + 1)^2 - 4 \cdot m \cdot (m + 1) \\ & \implies 
    m^2 + 2m + 1 - 4 \cdot (m^2 + m) \\ & \implies
    m^2 + 2m + 1 - 4m^2 - 4m \\ & \implies
    -3m^2 - 2m + 1
\end{align*}
Chegamos em uma equação do segundo grau. Então vamos calcular suas raízes e verificar $m$.
\begin{align*}
    m = \frac{2 \pm \sqrt{4 - (-12)}}{-6} & \implies
    \frac{2 \pm \sqrt{16}}{-6} \\ & \implies 
    \frac{2 \pm 4}{-6}
\end{align*}
Vejamos as possibilidades de $m$:
\begin{displaymath}
    m_1 = \frac{2 + 4}{-6} = -1 ; m_2 = \frac{2 - 4}{-6} = -\frac{1}{3}
\end{displaymath}
Vamos aplicar na equação encontrada no discriminante cada uma das possibilidades:
\begin{itemize}
    \item Se $m = -1$:
    \begin{align*}
        -3m^2 - 2m + 1 = 0 & \implies -3\cdot(-1)^2 - 2\cdot(-1) + 1 = 0 \\ & \implies 
        -3 + 2 + 1 = 0 \\ & \implies
        0 = 0
    \end{align*}
    Esta conclusão é verdadeira, então essa possibilidade para $m$ é válida.
    \item Se $m = -\frac{1}{3}$:
    \begin{align*}
        -3m^2 - 2m + 1 = 0 & \implies -3\cdot(-\frac{1}{3})^2 - 2\cdot(-\frac{1}{3}) + 1 = 0 \\ & \implies
        -1 + \frac{2}{3} + 1 = 0 \\ & \implies
        \frac{-3 + 2}{3} + 1 = 0 \\ & \implies
        \frac{-1}{3} + 1 = 0 \\ & \implies
        \frac{-1 + 3}{3} = 0 \\ & \implies
        \frac{2}{3} = 0
    \end{align*}
    Podemos concluir que tal resultado não é verdadeiro; logo, tal possibilidade de valor para $m$ não é válida.
\end{itemize}
Assim, para que a equação $mx^2 + (m + 1)x + (m + 1) = 0$ tenha somente uma raiz real, é necessário que $m = -1$.