\enquote{Determine o conjunto solução $S \subseteq \mathbb{Q}$ formado pelo(s) número(s) que, adicionado ao triplo de seu quadrado, resulta em 14}. 
\\
\emph{Resolução.} Podemos definir o conjunto $S \subseteq \mathbb{Q}$ como:
\begin{displaymath}
    S = \{x; x + 3x^2 = 14 \}
\end{displaymath}
Para descobrir quais os elementos de $S$ podemos resolver a equação $x + 3x^2 - 14$:
\begin{align*}
    \text{Discriminante} \\
    \Delta = b^2 - 4ac & =
    1 - 4 \cdot 3 \cdot (-14) \\ & =
    1 + 168 \\ & =
    169
    \\
    \text{Equação} \\
    x = \frac{-b \pm \sqrt{\Delta}}{2a} & \implies
    x = \frac{-1 \pm \sqrt{169}}{6} \\ & \implies
    x = \frac{-1 \pm 13}{6}
    \\
    \text{Raízes} \\
    x_1 &= \frac{-1 + 13}{6} = 2 \\ x_2 &= \frac{-1 - 13}{6} = -\frac{7}{3}
\end{align*}
Então temos que os elementos do conjunto $S$ são:
\begin{displaymath}
    S = \{ -\frac{7}{3}, 2 \}.
\end{displaymath}