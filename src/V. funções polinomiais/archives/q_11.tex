\enquote{Considere a função polinomial $p(x) = x^2 - 2$.}
\begin{enumerate}
    \item \enquote{Usando o Teorema de raízes racionais prove que $p(x)$ não
    possui raízes racionais}; \\
    \emph{Resolução.} Temos, pelo Teorema, que os candidatos às raízes de $p(x)$ são obtidos a partir dos divisores de:
    \begin{itemize}
        \item 2: $\pm 1, \pm 2$;
        \item 1: $\pm 1$
    \end{itemize}
    Podemos formar as frações $\frac{p}{q}$ a partir dos divisores; são elas $\pm \frac{1}{1}$ e $\pm \frac{2}{1}$. Vamos testar cada uma delas:
    \begin{align*}
        p(1) &= 1 - 2 = -1 \\
        p(-1) &= 1 - 2 = -1 \\
        p(2) &= 4 - 2 = 2 \\
        p(-2) &= 4 - 2 = 2
    \end{align*}
    Como não houve nenhum $p(\frac{p}{q}) = 0$, então $p$ não tem raízes racionais.
    \item \enquote{Mostre que $\sqrt{2}$ é raíz de $p(x)$;} \\
    \emph{Resolução.} Note que existe um número $\sqrt{2} \in \R$ tal que: 
    \begin{align*}
        p(\sqrt{2}) &= (\sqrt{2})^2 - 2 = 2 - 2 = 0
    \end{align*}
    Logo $\sqrt{2}$ é, de fato, raíz de $p$.
    \item \enquote{Conclua que $\sqrt{2}$ é irracional;} \\
    \emph{Resolução.} Sabendo que $p$ não tem raízes racionais e $\sqrt{2}$ é raíz de $p$, temos que $\sqrt{2}$ é irracional.
\end{enumerate}