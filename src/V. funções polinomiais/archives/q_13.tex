\enquote{Considere a função polinomial $p(x) = x^4 - x^3 - 2x^2 + 2x$. Utilize o Teorema das Raízes Racionais pelo menos uma vez para encontrar todas as raízes reais de $p(x)$. \\ 
\emph{Dica}: Antes de aplicar o Teorema das Raízes Racionais, identifique uma das raízes de $p(x)$ e fatore $p(x)$ para então usar o Teorema. Porque é preciso usar a dica ao invés de aplicar o Teorema diretamente em $p(x)$?} \\
\emph{Resolução.} Não podemos aplicar diretamente o Teorema sobre a função polinomial $p$ pois nela temos $a_0 = 0$, o que faria com que a única raíz do polinômio fosse 0. Desse modo, note que 0 é, de fato, uma raíz de $p$ visto que
\begin{displaymath}
    p(0) = 0 - 0 - 0 - 0 = 0
\end{displaymath}
Logo, temos:
\begin{align*}
    p(x) &= (x - 0) \cdot q(x) \\ &=
    (x - 0) \cdot (x^3 - x^2 - 2x + 2) \quad (\text{Divisão de polinômios})
\end{align*}
Aplicando o teorema para calcular as raízes de $q$, temos os divisores:
\begin{itemize}
    \item $a_0 (2): \pm 1, \pm 2$ \\
    \item $a_n (1): \pm 1$
\end{itemize}
Montando os números racionais, temos os candidatos $1, -1, 2, -2$. Vamos testar:
\begin{align*}
    p(1) &= 1 - 1 - 2 + 2 = 0 \\
    p(-1) &= -1 - 1 + 2 + 2 \ne 0 \\
    p(2) &= 8 - 4 - 4 + 2 \ne 0 \\
    p(-2) &= -8 -4 + 4 + 2 \ne 0
\end{align*}
Ademais, como existe uma raíz racional pra $q$, temos
\begin{align*}
    q(x) &= (x - 1) r(x) \\ &=
    (x - 1) (x^2 - 2) \quad (\text{Divisão de polinômios})
\end{align*}
Como raízes de $r$, temos:
\begin{align*}
    x^2 - 2 = 0 & \iff x^2 = 2 \iff x = \pm \sqrt{2}
\end{align*}
Por fim temos as raízes racionais de $p$ sendo 0 e 1, e as irracionais, $-\sqrt{2}$ e $\sqrt{2}$.