\enquote{Considere a função polinomial $p(x) = x^2 - q$, onde $q \in \N^*$ é um número primo.}
\begin{enumerate}
    \item \enquote{Usando o Teorema de raízes racionais prove que $p(x)$ não possui raízes racionais;} \\
    \emph{Resolução.}
    Pelo Teorema, temos como candidatos à raízes de $p$ os divisores de:
    \begin{itemize}
        \item $q: \pm 1, q$;
        \item 1: $\pm 1$,
    \end{itemize}
    Combinando cada um dos divisores, geramos os números racionais 1, -1, $\frac{1}{q}$ e $-\frac{1}{q}$. Podemos testá-los:
    \begin{align*}
        p(1) &= 1 - q \\
        p(-1) &= 1 - q \\
        p(\frac{1}{q}) &= \frac{1}{q^2} - q = \frac{1 - q^3}{q^2} \\
        p(-\frac{1}{q}) &= \frac{1}{q^2} - q = \frac{1 - q^3}{q^2} \\
    \end{align*}
    Portanto, não há raízes racionais para $p$.
    \item \enquote{Mostre que $\sqrt{q}$ é raíz de $p(x)$;} \\
    \emph{Resolução.} Note que, tomando $\sqrt{q} \in \R$, temos 
    \begin{displaymath}
        p(\sqrt{q}) = (\sqrt{q})^2 - q = q - q = 0        
    \end{displaymath}
    Logo $\sqrt{q}$ é, de fato, raíz de $p$.
    \item \enquote{Conclua que $\sqrt{q}$ é irracional;} \\
    \emph{Resolução.} Como $p$ não tem raízes racionais e $\sqrt{q}$ é raíz de $p$, temos que $\sqrt{q}$ é irracional.
\end{enumerate}