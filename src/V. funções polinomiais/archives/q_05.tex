\enquote{Para cada uma das funções quadráticas abaixo, escreva-a na forma $a(x - m)^2 + k$. A seguir, calcule suas raízes (se existirem), o eixo de simetria de seu gráfico e seu valor mínimo e máximo.}
\begin{enumerate}
    \item $f(x) = x^2 - 8x + 23$ \\
    \emph{Resolução.}
    \begin{align*}
        x^2 - 8x + 23 & = x^2 - 8x + 16 + 7 \\ & =
        (x - 4)^2 + 7
    \end{align*}
    Note que a função não tem raízes reais visto que:
    \begin{align*}
        \Delta = 64 - 92 = -28
    \end{align*}
    Seja $x \in \R$ o eixo de simetria tal que
    \begin{align*}
        x = -\frac{-8}{2} = 4
    \end{align*}
    Como $a > 0$, a função não tem valor máximo. Contudo, existe um valor mínimo $y_0 \in \R$ tal que
    \begin{align*}
        y_0 = f(4) = (4 - 4)^2 + 7 = 7
    \end{align*}
    Portanto, o valor mínimo da função é 4 e o ponto (4, 7) é o ponto de mínimo.

    \item $f(x) = 8x - 2x^2$. \\
    \emph{Resolução.}
    \begin{align*}
        8x -2x^2 = -2x(x - 4)
    \end{align*}
    Temos que suas raízes são 0 e 4 pois são os valores que zeram a expressão e geram $y = 0$. \\
    Seja $x \in \R$ o eixo de simetria tal que
    \begin{align*}
        x = -\frac{8}{-4} = 2
    \end{align*}
    Como $a < 0$, a função não tem valor mínimo. Contudo, existe um ponto máximo $y_1 \in \R$ tal que
    \begin{align*}
        y_1 = f(2) = 8 \cdot 2 - 2 \cdot 4 = 8
    \end{align*}
    Portanto, o valor máximo da função é 2 e o ponto (2, 8) é o ponto de máximo absoluto.

    \item $f(x) = 2x^2 - 16x + 46$ \\
    \emph{Resolução.}
    \begin{align*}
        2x^2 - 16x + 46 &= 2x^2 - 16x + 64 - 18 \\ 
        &= 2(x-8)^2 - 18
    \end{align*}
    Note que a função não tem raízes reais visto que:
    \begin{align*}
        \Delta = 256 - 368 = -112
    \end{align*}
    Seja $x \in \R$ o eixo de simetria tal que
    \begin{align*}
        x = -\frac{-16}{4} = 4
    \end{align*}
    Como $a > 0$, a função não tem valor máximo. Contudo, existe um ponto mínimo $y_2 \in \R$ tal que
    \begin{align*}
        y_2 = f(4) = 2 \cdot 16 - 16 \cdot 4 + 46 = 14 
    \end{align*}
    Portanto, o valor mínimo da função é 4 e o ponto (4, 14) é o ponto de mínimo absoluto.
\end{enumerate}