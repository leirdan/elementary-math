\enquote{Considere a função $f: ]0; 1[ \rightarrow \R$ tal que
\begin{displaymath}
    f(x) = \left\{\begin{aligned}
        \frac{1}{x} - 2&; \quad \text{se $x \le \frac{1}{2}$} \\
        2 - \frac{1}{1 - x}&; \quad \text{se $x > \frac{1}{2}$}
    \end{aligned}\\ 
    \right.
\end{displaymath}
Mostre que $f$ é bijetiva.} \\

\emph{Resolução.} Provemos sua injetividade para cada caso:
\begin{itemize}
    \item Se $x \le \frac{1}{2}$; \\
    Sejam $a, b \in ]0; 1[$ tal que
    \begin{align*}
        f(a) = f(b) & \implies \frac{1}{a} - 2 = \frac{1}{b} -2 \\ & \implies
        \frac{1}{a} = \frac{1}{b} \\ & \implies
        a = b
    \end{align*}
    \item Se $x > \frac{1}{2}$; \\
    Sejam $a, b \in ]0; 1[$ tal que 
    \begin{align*}
        f(a) = f(b) & \implies 2 - \frac{1}{1 - a} = 2 - \frac{1}{1 - b} \\ & \implies
        - \frac{1}{1 - a} = - \frac{1}{1 - b} \\ & \implies
        \frac{1}{1 - a} = \frac{1}{1 - b} \\ & \implies
        1 - a = 1 - b \\ & \implies
        a = b
    \end{align*}
    \item Se $x_1 \le \frac{1}{2}$ e $x_2 > \frac{1}{2}$; \\
    Sejam $a, b$ tais que:
    \begin{align*}
        0 < a \le \frac{1}{2} & \implies \frac{1}{a} \ge \frac{1}{\frac{1}{2}} \\ & \implies
        \frac{1}{a} \ge 2 \\ & \implies
        \frac{1}{a} - 2 \ge 0 \\ & \implies
        f(a) \ge 0
    \end{align*}
    e
    \begin{align*}
        1 > b > \frac{1}{2} & \implies 0 > b - 1 > -\frac{1}{2} \\ & \implies
        1 - b < \frac{1}{2} \\ & \implies
        \frac{1}{1 - b} > 2 \\ & \implies
        -\frac{1}{1 - b} < -2 \\ & \implies
        2 - \frac{1}{1 - b} < 0 \\ & \implies
        f(b) < 0
    \end{align*}
    Assim, $a \ne b \implies f(a) \ne f(b)$.
\end{itemize}
Provada sua injetividade, provemos também sua sobrejetividade.
\begin{itemize}
    \item Se $x \le \frac{1}{2}$; \\
    Seja $y \in \R_+$. Note que:
    \begin{align*}
        y \ge 0 & \implies y + 2 \ge 0 \\ & \implies
        \frac{1}{y + 2} \le \frac{1}{2}
    \end{align*}
    e 
    \begin{align*}
        1 > 0 & \implies \frac{1}{y + 2} > 0
    \end{align*}
    Portanto, 
    \begin{displaymath} 
        0 < \frac{1}{y + 2} \le \frac{1}{2}
    \end{displaymath}
    Logo, existe um número $\frac{1}{y + 2} \in ]0; 1[$ tal que
    \begin{align*}
        f(\frac{1}{y + 2}) & = \frac{1}{\frac{1}{y + 2}} - 2 \\ & =
        y + 2 - 2 \\ & =
        y
    \end{align*}
    Temos que a função é sobrejetiva para $x \le \frac{1}{2}$.
    \item Se $x > \frac{1}{2}$; \\
    Seja $y \in \R_-^*$. Note que:
    \begin{align*}
        y < 0 & \implies -y > 0 \\ & \implies
        -y + 2 > 2 \\ & \implies
        \frac{1}{-y + 2} < \frac{1}{2} \\ & \implies
        -\frac{1}{-y + 2} > -\frac{1}{2} \\ & \implies
        1 - \frac{1}{-y + 2} > \frac{1}{2} 
    \end{align*}
    e
    \begin{align*}
        1 > 0 & \implies
        \frac{1}{-y + 2} > 0 \\ & \implies
        -\frac{1}{-y + 2} < 0 \\ & \implies
        1 - \frac{1}{-y + 2} < 1
    \end{align*}
    Portanto, 
    \begin{displaymath}
        \frac{1}{2} < 1 - \frac{1}{-y + 2} < 1,
    \end{displaymath}
    Logo, existe um número $1 - \frac{1}{-y + 2} \in ]0; 1[$ tal que
    \begin{align*}
        f(1 - \frac{1}{-y + 2}) & = 2 - \frac{1}{1 - (1 - \frac{1}{-y + 2})} \\ & =
        2 - \frac{1}{\frac{1}{-y + 2}} \\ & =
        2 - (-y + 2) \\ & =
        y
    \end{align*} 
    Temos que a função é sobrejetiva para $x > \frac{1}{2}$.
\end{itemize}
Assim, com a sobrejetividade provada para ambos os casos, temos que $f$ é sobrejetiva.