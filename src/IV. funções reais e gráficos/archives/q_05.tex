\enquote{Seja $f: \R \rightarrow \R$ a função cuja lei de associação é da forma abaixo:
    \begin{displaymath}
        f(x) = \left\{\begin{aligned}
            x^2 + 3x&; \quad \text{se $x \ge 0$} \\
            \frac{3}{2}x&; \quad \text{se $x < 0$}
        \end{aligned}\\ 
        \right.
    \end{displaymath}
Mostre que $f$ é bijetiva.} \\
\emph{Resolução.} Provemos que é injetiva.
\begin{itemize}
    \item Se $a, b \ge 0$; \\
    Sejam $a, b \in \R$. Temos que:
    \begin{align*}
        f(a) = f(b) & \implies
        a^2 + 3a = b^2 + 3b \\ & \implies
        a^2 + 3a + (\frac{3}{2})^2 = b^2 + 3b + (\frac{3}{2})^2 \\ & \implies
        (a + \frac{3}{2})^2 = (b + \frac{3}{2})^2 \\ & \implies 
        \sqrt{(a + \frac{3}{2})^2} = \sqrt{(b + \frac{3}{2})^2} \\ & \implies
        |a + \frac{3}{2}| = |b + \frac{3}{2}| \\ & \implies
        a + \frac{3}{2} = b + \frac{3}{2} \quad (a, b \ge 0) \\ & \implies
        a = b
    \end{align*}
    \item Se $a, b < 0$; \\
    Sejam $a, b \in \R$. Temos que:
    \begin{align*}
        f(a) = f(b) & \implies \frac{3}{2}a = \frac{3}{2}b \\ & \implies
        \frac{2}{3} \cdot \frac{3}{2}a = \frac{2}{3} \cdot \frac{3}{2}b \\ & \implies
        a = b
    \end{align*}
    \item Se $a \ge 0$ e $b < 0$; \\
    Sejam $a, b \in \R$. Temos:
    \begin{align*}
        a \ge 0 & \implies a + \frac{3}{2} \ge \frac{3}{2} \\ & \implies
        (a + \frac{3}{2})^2 \ge (\frac{3}{2})^2 \\ & \implies
        a^2 + 3a + \frac{9}{4} \ge \frac{9}{4} \\ & \implies
        a^2 + 3a \ge 0 \\ & \implies
        f(a) \ge 0
    \end{align*}
    Também temos que:
    \begin{align*}
        b < 0 & \implies \frac{3}{2}b < 0 \\ & \implies
        f(b) < 0
    \end{align*}
    Assim, $a \ne b \implies f(a) \ne f(b)$.
\end{itemize}
Logo, a função é injetiva para todos os casos. Provemos agora que é sobrejetiva.
\begin{itemize}
    \item Se $x \ge 0$; \\
    Seja $y \in \R_+$, ou seja:
    \begin{align*}
        y \ge 0 & \implies y + \frac{9}{4} \ge \frac{9}{4} \\ & \implies
        \sqrt{y + \frac{9}{4}} \ge \frac{3}{2} \\ & \implies 
        \sqrt{y + \frac{9}{4}} - \frac{3}{2} \ge 0
    \end{align*} 
    Logo, existe um número $\sqrt{y + \frac{9}{4}} - \frac{3}{2} \in \R$ tal que
    \begin{align*}
        f(\sqrt{y + \frac{9}{4}} - \frac{3}{2}) & = (\sqrt{y + \frac{9}{4}} - \frac{3}{2})^2 + 3 \cdot (\sqrt{y + \frac{9}{4}} - \frac{3}{2}) \\ & =
        y + \frac{9}{4} - 3 \cdot \sqrt{y + \frac{9}{4}} + \frac{9}{4} + 3\sqrt{y + \frac{9}{4}} - \frac{9}{2} \\ & =
        y + \frac{18}{4} - \frac{9}{2} \\ & =
        y
    \end{align*}  
    Portanto, a função é sobrejetiva para quaisquer $x \in \R_+$.  
    \item Se $x < 0$; \\
    Seja $y \in \R_-^*$. Note que:
    \begin{align*}
        y < 0 & \implies \frac{2}{3}y < 0
    \end{align*}
    Logo, existe um número $\frac{2}{3}y \in \R_-^*$ tal que
    \begin{align*}
        f(\frac{2}{3}y) & = \frac{3}{2}x \\ &= \frac{3}{2} \cdot \frac{2}{3}y \\ & = y
    \end{align*}
    A função é, portanto, sobrejetiva para quaisquer $x \in \R_-^*$.
\end{itemize}
Como $f$ é sobrejetiva para ambos os casos temos que a função é, de fato, sobrejetiva para qualquer $x \in \R$.