\enquote{Seja $f : \N \rightarrow \R$ e $g : \R \rightarrow \N$. Determine se as afirmações
abaixo são verdadeiras ou falsas, justificando suas respostas.
As funções que forem usadas como contraexemplo podem ser
exibidas somente com o esboço de seu gráfico.}
\begin{enumerate}
    \item \enquote{A função $g$ pode ser ilimitada inferiormente;} \\
    \emph{Falso.} Note que $g(x) > 0$ para todo $x \in \R$. Logo, a função não é ilimitada inferiormente, podendo seu limite ser 0, por exemplo.
    \item \enquote{$f$ é limitada superiormente ou $f$ é limitada inferiormente;} \\
    \emph{Falso.} Seja
    \begin{displaymath}
        f(x) = \left\{
            \begin{aligned}
                x^2 &, \text{se $x$ é par}, \\
                -x &, \text{se $x$ é ímpar}
            \end{aligned}
         \right.
    \end{displaymath}
    Note que não haverá limite inferior nem superior para essa função.
\end{enumerate}