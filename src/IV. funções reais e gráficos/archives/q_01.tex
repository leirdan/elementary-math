\enquote{Em cada um dos itens abaixo, defina uma função com a lei de formação dada (indicando domínio e contradomínio). Verifique se é injetiva, sobrejetiva ou bijetiva, a função:}
\begin{enumerate}
    \item \enquote{Que a cada ponto do plano cartesiano associa a distância desse plano à origem do plano;} 
    \\ \emph{Resolução}. Seja $A$ a matriz definida acima. Calculemos seu determinante a partir de um cofator:
\begin{align*}
    &\left|\begin{array}{ccc}
         0 & 3 & 1 \\
         1 & 1 & 2 \\
         3 & 2 & 4
    \end{array}\right| \quad (c_3 \rightarrow c_3 - 2c_2) \\ =
    &\left|\begin{array}{ccc}
         0 & 3 & -5 \\
         1 & 1 & 0 \\
         3 & 2 & 0
    \end{array}\right|
\end{align*}
Fixada a coluna 3, teremos:
\begin{align*}
    \det A =& a_{13} \cdot C_{13} + a_{23} \cdot C_{23} + a_{33} \cdot C_{33} \\ =&
    -5 \cdot C_{13} + 0 \cdot C_{23} + 0 \cdot  C_{33} \\ =&
    -5 \cdot (-1)^{3 + 3} \cdot 
    \left|\begin{array}{cc}
        1&1 \\
        3&2
    \end{array} \right| \\ =&
    -5 \cdot 1 \cdot -1 \\ =&
    5
\end{align*}
Como $\det A \ne 0$, então vamos calcular a matriz inversa:
\begin{align*}
    &\left[\begin{array}{cccccc}
         0 & 3 & 1 & 1 & 0 & 0 \\
         1 & 1 & 2 & 0 & 1 & 0 \\
         3 & 2 & 4 & 0 & 0 & 1
    \end{array}\right] \quad (l_2 \leftrightarrow l_1) \\ \rightarrow
    &\left[\begin{array}{cccccc}
         1 & 1 & 2 & 0 & 1 & 0 \\
         0 & 3 & 1 & 1 & 0 & 0 \\
         3 & 2 & 4 & 0 & 0 & 1
    \end{array}\right]  \quad (l_3 \rightarrow l_3 - 3l_1) \\ \rightarrow
    &\left[\begin{array}{cccccc}
         1 & 1 & 2 & 0 & 1 & 0 \\
         0 & 3 & 1 & 1 & 0 & 0 \\
         0 & -1 & -2 & 0 & -3 & 1
    \end{array}\right] \quad (l_3 \leftrightarrow l_2) \\ \rightarrow
    &\left[\begin{array}{cccccc}
         1 & 1 & 2 & 0 & 1 & 0 \\
         0 & -1 & -2 & 0 & -3 & 1 \\
         0 & 3 & 1 & 1 & 0 & 0  
    \end{array}\right] \quad (l_1 \rightarrow l_1 + l_2; l_3 \rightarrow l_3 + 3l_2) \\ \rightarrow
    &\left[\begin{array}{cccccc}
         1 & 0 & 0 & 0 & -2 & 1 \\
         0 & -1 & -2 & 0 & -3 & 1 \\
         0 & 0 & -5 & 1 & -9 & 3  
    \end{array}\right] \quad (l_2 \rightarrow -l_2; l_3 \rightarrow -l_3) \\ \rightarrow
    &\left[\begin{array}{cccccc}
         1 & 0 & 0 & 0 & -2 & 1 \\
         0 & 1 & 2 & 0 & 3 & -1 \\
         0 & 0 & 5 & -1 & 9 & -3  
    \end{array}\right] \quad (l_2 \rightarrow l_2 - \frac{2}{5}l_3) \\ \rightarrow
    &\left[\begin{array}{cccccc}
         1 & 0 & 0 & 0 & -2 & 1 \\
         0 & 1 & 0 & \frac{2}{5} & -\frac{3}{5} & \frac{1}{5} \\
         0 & 0 & 5 & -1 & 9 & -3  
    \end{array}\right] \quad (l_3 \rightarrow \frac{l_3}{5}) \\ \rightarrow
    &\left[\begin{array}{cccccc}
         1 & 0 & 0 & 0 & -2 & 1 \\
         0 & 1 & 0 & \frac{2}{5} & -\frac{3}{5} & \frac{1}{5} \\
         0 & 0 & 1 & -\frac{1}{5} & \frac{9}{5} & -\frac{3}{5}  
    \end{array}\right] 
\end{align*}
Logo, a matriz inversa $A^{-1}$ é
\begin{displaymath}
    \left[\begin{array}{ccc}
         0 & -2 & 1 \\
        \frac{2}{5} & -\frac{3}{5} & \frac{1}{5} \\
        -\frac{1}{5} & \frac{9}{5} & -\frac{3}{5}  
    \end{array}\right] 
\end{displaymath} 
    \item \enquote{Que a cada dois números naturais associa seu MDC;} 
    \\ \emph{Resolução.} Tomando a matriz acima como $B$, calculemos $\det B$ com cofatores:
	\begin{align*}
        &\left|\begin{array}{ccc}
            3 & -6 & 9 \\
            -2 & 7 & -2 \\
            0 & 1 & 5
            \end{array}\right| \quad (l_1 \rightarrow \frac{l_1}{3}) \\ =
        & 3 \cdot \left|\begin{array}{ccc}
            1 & -2 & 3 \\
            -2 & 7 & -2 \\
            0 & 1 & 5
            \end{array}\right| (l_2 \rightarrow l_2 + 2l_1) \\ =
        & 3 \cdot \left|\begin{array}{ccc}
            1 & -2 & 3 \\
            0 & 3 & 4 \\
            0 & 1 & 5
            \end{array}\right|
    \end{align*}
Fixando a coluna 1 temos:
\begin{align*}
    \det B &= 3 \cdot (a_{11} \cdot C_{11} + a_{21} \cdot C_{21} + a_{31}\cdot C_{31}) \\ &=
    3 \cdot (1 \cdot C_{11} + 0 + 0) \\ &=
    3 \cdot (1 \cdot (-1)^{1+1} \cdot 
	\left|\begin{array}{cc} 
		3 & 4 \\ 1 & 5 
	\end{array} \right|) \\ & =
	3 \cdot (1 \cdot 1 \cdot 11) \\ & = 
	33
\end{align*}
Concluímos que $\det B = 33$. Calculemos sua inversa:
\begin{align*}
    &\left[\begin{array}{cccccc}
	3 & -6 & 9 & 1 & 0 & 0 \\
	-2 & 7 & -2 & 0 & 1 & 0\\
	0 & 1 & 5 & 0 & 0 & 1
	\end{array}\right] \quad (l_1 \rightarrow \frac{l_1}{3}) \\ \rightarrow
    &\left[\begin{array}{cccccc}
	1 & -2 & 3 & \frac{1}{3} & 0 & 0 \\
	-2 & 7 & -2 & 0 & 1 & 0\\
	0 & 1 & 5 & 0 & 0 & 1
	\end{array}\right] \quad (l_2 \rightarrow l_2 + 2l_1) \\ \rightarrow
    &\left[\begin{array}{cccccc}
	1 & -2 & 3 & \frac{1}{3} & 0 & 0 \\
	0 & 3 & 4 & \frac{2}{3} & 1 & 0\\
	0 & 1 & 5 & 0 & 0 & 1
	\end{array}\right] \quad (l_2 \leftrightarrow l_3) \\ \rightarrow
    &\left[\begin{array}{cccccc}
	1 & -2 & 3 & \frac{1}{3} & 0 & 0 \\
	0 & 1 & 5 & 0 & 0 & 1\\
	0 & 3 & 4 & \frac{2}{3} & 1 & 0
	\end{array}\right] \quad (l_1 \rightarrow l_1 + 2l_2; l_3 \rightarrow l_3 - 3l_2) \\ \rightarrow
    &\left[\begin{array}{cccccc}
	1 & 0 & 13 & \frac{1}{3} & 0 & 2 \\
	0 & 1 & 5 & 0 & 0 & 1 \\
	0 & 0 & -11 & \frac{2}{3} & 1 & -3
	\end{array}\right] \quad (l_2 \rightarrow l_2 + \frac{5}{11}l_3) \\ \rightarrow
    &\left[\begin{array}{cccccc}
	1 & 0 & 13 & \frac{1}{3} & 0 & 2 \\
	0 & 1 & 0 & \frac{10}{33} & \frac{5}{11} & -\frac{4}{11} \\
	0 & 0 & -11 & \frac{2}{3} & 1 & -3
	\end{array}\right] \quad (l_3 \rightarrow -\frac{l_3}{11}) \\ \rightarrow
    &\left[\begin{array}{cccccc}
	1 & 0 & 13 & \frac{1}{3} & 0 & 2 \\
	0 & 1 & 0 & \frac{10}{33} & \frac{5}{11} & -\frac{4}{11} \\
	0 & 0 & 1 & -\frac{2}{33} & -\frac{1}{11} & \frac{3}{11}
	\end{array}\right] \quad (l_1 \rightarrow l_1 - 13l_3) \\ \rightarrow
    &\left[\begin{array}{cccccc}
	1 & 0 & 0 & \frac{37}{33} & \frac{13}{11} & -\frac{17}{11} \\
	0 & 1 & 0 & \frac{10}{33} & \frac{5}{11} & -\frac{4}{11} \\
	0 & 0 & 1 & -\frac{2}{33} & -\frac{1}{11} & \frac{3}{11}
    \end{array}\right]
\end{align*}
Então, concluímos que a matriz inversa $B^{-1}$ é
\begin{align*}
    \left[\begin{array}{ccc}
         \frac{37}{33} & \frac{13}{11} & -\frac{17}{11}\\
         \frac{10}{33} & \frac{5}{11} & -\frac{4}{11} \\
         -\frac{2}{33} & -\frac{1}{11} & \frac{3}{11}
    \end{array}\right]
\end{align*} 
    \item \enquote{Que a cada polinômio (não nulo) com coeficientes reais associa seu grau;} 
    \\ \emph{Resolução.} Tomando a matriz acima como $D$, calculemos $\det D$ com cofatores:
\begin{align*}
    &\left|\begin{array}{ccccc}
        4 & 0 & 0 & 1 & 0 \\
        3 & 3 & 3 & -1 & 0 \\
        1 & 2 & 4 & 2 & 3 \\
        9 & 4 & 6 & 2 & 3 \\
        2 & 2 & 4 & 2 & 3 \\
    \end{array}\right| \quad (l_5 \rightarrow l_5 - l_3) = \\
    &\left|\begin{array}{ccccc}
        4 & 0 & 0 & 1 & 0 \\
        3 & 3 & 3 & -1 & 0 \\
        1 & 2 & 4 & 2 & 3 \\
        9 & 4 & 6 & 2 & 3 \\
        1 & 0 & 0 & 0 & 0 \\
    \end{array}\right|
\end{align*}
Fixando a linha 5 temos:
\begin{align*}
    \det D &= a_{51} \cdot C_{51} + a_{52} \cdot C_{52} + a_{53} \cdot C_{53} + a_{54} \cdot C_{54} + a_{55} \cdot C_{55} \\ &=
    a_{51} \cdot (-1)^{5 + 1} \cdot \det \left|\begin{array}{cccc}
         0 & 0 & 1 & 0  \\
         3 & 3 & -1 & 0 \\
         2 & 4 & 2 & 3 \\
         4 & 6 & 2 & 3 \\
    \end{array} \right|
\end{align*}
Na matriz menor $D_{51}$ fixamos a linha 1 e obtemos:
\begin{align*}
    \det D_{51} &= a_{11} \cdot C_{11} + a_{12} \cdot C_{12} + a_{13} \cdot C_{13} + a_{14} \cdot C_{14} \\ &=
    a_{13} \cdot (-1)^{1 + 3} \cdot \det 
    \left|\begin{array}{ccc}
         3 & 3& 0 \\
         2 & 4& 3 \\
         4 & 6 & 3
    \end{array}\right| \\ &=
    1 \cdot 1 \cdot [(36 + 36 + 0) - (0 + 18 + 54)] \\ &=
    0
\end{align*}
Aplicando o resultado de $\det D_{51}$ no cálculo de $\det D$ teremos como resultado final 0. Não há, portanto, matriz inversa de $D$. 
    \item \enquote{Que a cada figura plana fechada e limitada associa a sua área;} 
    \\ \input{../../IV. funções reais e gráficos/archives/q_01/item_d.tex} 
    \item \enquote{Que a cada subconjunto de $\R$ associa seu complementar;} 
    \\ \emph{Resolução}: Seja a função
\begin{align*}
    c: \mathbb{R} &\rightarrow \mathbb{R} \\
    R &\rightarrow R^C
\end{align*}
\par Injetividade: \textbf{sim}. Sejam $A, B \subseteq \mathbb{R}$, de modo que $A \ne B$. Então temos:
    \begin{displaymath}
        A \ne B \implies c(A) \ne c(B) \implies A^C \ne B^C
    \end{displaymath}
    Portanto é injetiva.
\par Sobrejetiva: \textbf{sim}. Seja $N^C \subseteq \mathbb{R}$; logo, existe $N \subseteq \mathbb{R}$ de modo que
    \begin{displaymath}
        c(N) = N^C
    \end{displaymath}
    Portanto é sobrejetiva e, além disso, é também bijetiva. 
    \item \enquote{Que a cada subconjunto finito de $\N$ associa seu número de elementos;} 
    \\ \emph{Resolução}: Tomando $n$ como uma função que calcula a quantidade de elementos de um conjunto, seja a função
\begin{align*}
    h: \mathbb{N} &\rightarrow \mathbb{N} \\
    X &\rightarrow n(X)
\end{align*}
\par Injetividade: \textbf{não}. Tomando $A = \{ 1, 2, 3\}$ e $B = \{6, 7, 8\}$, temos 
    \begin{displaymath}
    h(A) = 3 = h(B)
    \end{displaymath}
\par Sobrejetividade: \textbf{sim}. Seja $n \in \mathbb{N}$; temos um conjunto $X \subseteq \mathbb{N}$:
    \begin{displaymath}
        h(X) \implies n(X) = n
    \end{displaymath}
Portanto é sobrejetiva. 
    \item \enquote{Que a cada subconjunto não vazio de $\N$ associa seu menor elemento;} 
    \\ \emph{Resolução}: Tomando $inf$ como uma função que calcula o menor elemento de um conjunto, seja a função
\begin{align*}
    i: \mathbb{N}\backslash \emptyset &\rightarrow \mathbb{N} \\
    X &\rightarrow inf(X)
\end{align*}
\par Injetividade: \textbf{não}. Sendo os conjuntos $A = \{ 1, 3, 5, 7\}$ e $B = \{ 1, 2, 4, 8 \}$, note que
    \begin{displaymath}
        i(A) = 1 = i(B)
    \end{displaymath}
\par Sobrejetividade: \textbf{sim}. Tome $C \subseteq \mathbb{N}$ e $n \in C$; então, temos:
    \begin{displaymath}
        i(C) \implies inf(C) = n
    \end{displaymath}
    Portanto é sobrejetiva. 
\end{enumerate}