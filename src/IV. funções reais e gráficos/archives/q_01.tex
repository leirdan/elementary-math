\enquote{Em cada um dos itens abaixo, defina uma função com a lei de formação dada (indicando domínio e contradomínio). Verifique se é injetiva, sobrejetiva ou bijetiva, a função:}
\begin{enumerate}
    \item \enquote{Que a cada ponto do plano cartesiano associa a distância desse plano à origem do plano;} 
    \\ \emph{Resolução}. Seja a função
\begin{align*}
    d: \mathbb{R}^2 &\rightarrow \mathbb{R}_+ \\
    (x, y) &\rightarrow \sqrt{(x - 0) + (y - 0)}
\end{align*}
\par Injetividade: \textbf{não}. Sejam os pontos $(6, 4), (4,6) \in \mathbb{R}$ e note que:
    \begin{align*}
        f(6,4) = \sqrt{10} = f(4,6)
    \end{align*}
\par Sobrejetividade: \textbf{sim}. Tomando $\sqrt{x + y} \in \mathbb{R}_+$ e $x, y \in \mathbb{R}$, temos:
    \begin{align*}
        f(x, y) = \sqrt{(x - 0) + (y - 0)} = \sqrt{x + y}
    \end{align*} 
    \item \enquote{Que a cada dois números naturais associa seu MDC;} 
    \\ \emph{Resolução}: Tomando $mdc$ como uma função que calcula o máximo divisor comum entre dois números, seja a função 
\begin{align*}
    f: \mathbb{N}^2 &\rightarrow \mathbb{N}^* \\
   a, b &\rightarrow mdc(a, b).
\end{align*} 
\par Injetividade: \textbf{não}, visto que 
    \begin{displaymath}
        f(6, 2) = 2 = f(4, 2)
    \end{displaymath}
\par Sobrejetividade: \textbf{sim}. Tomando $x \in \N^*$ e $a, b \in \N$, temos:
    \begin{displaymath}
        f(a, b) \implies mdc(a, b) = x.
    \end{displaymath}
    Como é válido para qualquer $x \in \mathbb{N}^*$ temos que a função é sobrejetiva. 
    \item \enquote{Que a cada polinômio (não nulo) com coeficientes reais associa seu grau;} 
    \\ \emph{Resolução.} Tomando a matriz acima como $D$, calculemos $\det D$ com cofatores:
\begin{align*}
    &\left|\begin{array}{ccccc}
        4 & 0 & 0 & 1 & 0 \\
        3 & 3 & 3 & -1 & 0 \\
        1 & 2 & 4 & 2 & 3 \\
        9 & 4 & 6 & 2 & 3 \\
        2 & 2 & 4 & 2 & 3 \\
    \end{array}\right| \quad (l_5 \rightarrow l_5 - l_3) = \\
    &\left|\begin{array}{ccccc}
        4 & 0 & 0 & 1 & 0 \\
        3 & 3 & 3 & -1 & 0 \\
        1 & 2 & 4 & 2 & 3 \\
        9 & 4 & 6 & 2 & 3 \\
        1 & 0 & 0 & 0 & 0 \\
    \end{array}\right|
\end{align*}
Fixando a linha 5 temos:
\begin{align*}
    \det D &= a_{51} \cdot C_{51} + a_{52} \cdot C_{52} + a_{53} \cdot C_{53} + a_{54} \cdot C_{54} + a_{55} \cdot C_{55} \\ &=
    a_{51} \cdot (-1)^{5 + 1} \cdot \det \left|\begin{array}{cccc}
         0 & 0 & 1 & 0  \\
         3 & 3 & -1 & 0 \\
         2 & 4 & 2 & 3 \\
         4 & 6 & 2 & 3 \\
    \end{array} \right|
\end{align*}
Na matriz menor $D_{51}$ fixamos a linha 1 e obtemos:
\begin{align*}
    \det D_{51} &= a_{11} \cdot C_{11} + a_{12} \cdot C_{12} + a_{13} \cdot C_{13} + a_{14} \cdot C_{14} \\ &=
    a_{13} \cdot (-1)^{1 + 3} \cdot \det 
    \left|\begin{array}{ccc}
         3 & 3& 0 \\
         2 & 4& 3 \\
         4 & 6 & 3
    \end{array}\right| \\ &=
    1 \cdot 1 \cdot [(36 + 36 + 0) - (0 + 18 + 54)] \\ &=
    0
\end{align*}
Aplicando o resultado de $\det D_{51}$ no cálculo de $\det D$ teremos como resultado final 0. Não há, portanto, matriz inversa de $D$. 
    \item \enquote{Que a cada figura plana fechada e limitada associa a sua área;} 
    \\ Não fiz ainda. 
    \item \enquote{Que a cada subconjunto de $\R$ associa seu complementar;} 
    \\ \emph{Resolução}: Seja a função
\begin{align*}
    c: \mathbb{R} &\rightarrow \mathbb{R} \\
    R &\rightarrow R^C
\end{align*}
\par Injetividade: \textbf{sim}. Sejam $A, B \subseteq \mathbb{R}$, de modo que $A \ne B$. Então temos:
    \begin{displaymath}
        A \ne B \implies c(A) \ne c(B) \implies A^C \ne B^C
    \end{displaymath}
    Portanto é injetiva.
\par Sobrejetiva: \textbf{sim}. Seja $N^C \subseteq \mathbb{R}$; logo, existe $N \subseteq \mathbb{R}$ de modo que
    \begin{displaymath}
        c(N) = N^C
    \end{displaymath}
    Portanto é sobrejetiva e, além disso, é também bijetiva. 
    \item \enquote{Que a cada subconjunto finito de $\N$ associa seu número de elementos;} 
    \\ \emph{Resolução}: Tomando $n$ como uma função que calcula a quantidade de elementos de um conjunto, seja a função
\begin{align*}
    h: \mathbb{N} &\rightarrow \mathbb{N} \\
    X &\rightarrow n(X)
\end{align*}
\par Injetividade: \textbf{não}. Tomando $A = \{ 1, 2, 3\}$ e $B = \{6, 7, 8\}$, temos 
    \begin{displaymath}
    h(A) = 3 = h(B)
    \end{displaymath}
\par Sobrejetividade: \textbf{sim}. Seja $n \in \mathbb{N}$; temos um conjunto $X \subseteq \mathbb{N}$:
    \begin{displaymath}
        h(X) \implies n(X) = n
    \end{displaymath}
Portanto é sobrejetiva. 
    \item \enquote{Que a cada subconjunto não vazio de $\N$ associa seu menor elemento;} 
    \\ \emph{Resolução}: Tomando $inf$ como uma função que calcula o menor elemento de um conjunto, seja a função
\begin{align*}
    i: \mathbb{N}\backslash \emptyset &\rightarrow \mathbb{N} \\
    X &\rightarrow inf(X)
\end{align*}
\par Injetividade: \textbf{não}. Sendo os conjuntos $A = \{ 1, 3, 5, 7\}$ e $B = \{ 1, 2, 4, 8 \}$, note que
    \begin{displaymath}
        i(A) = 1 = i(B)
    \end{displaymath}
\par Sobrejetividade: \textbf{sim}. Tome $C \subseteq \mathbb{N}$ e $n \in C$; então, temos:
    \begin{displaymath}
        i(C) \implies inf(C) = n
    \end{displaymath}
    Portanto é sobrejetiva. 
\end{enumerate}