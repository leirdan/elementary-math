\enquote{Sejam $f: \R \rightarrow \R$. Determine se as afirmações abaixo são
verdadeiras ou falsas, justificando suas respostas. As funções
que forem usadas como contraexemplo podem ser exibidas somente com o esboço do gráfico de sua função.}
\begin{enumerate}
    \item Se $f$ é limitada superiormente, então $f$ tem pelo menos um máximo absoluto. \\
    \emph{Resolução.} Afirmação verdadeira. $f$ é limitada superiormente se existe um número $M \in \R$ tal que
    \begin{align*}
        M \ge f(x)
    \end{align*}
    para qualquer $x$.
    Além disso, um máximo absoluto é um número $x_0 \in D$ tal que
    \begin{align*}
        f(x_0) \ge f(x)
    \end{align*} 
    para qualquer $x$.
\end{enumerate}