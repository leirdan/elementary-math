\enquote{Considere a função $f: \N^* \rightarrow \Z$ tal que
\begin{align*}
f(n) = \left\{
        \begin{aligned}
            \frac{-n}{2} &; \quad \text{se $n$ é par} \\
            \frac{n - 1}{2} &; \quad \text{se $n$ é ímpar}
        \end{aligned}
        \right.
\end{align*}
Mostre que $f$ é bijetiva.} \\
\emph{Resolução}: Devemos mostrar que $f$ é injetiva e sobrejetiva para todos os casos. Começamos com a injetividade para:
\begin{itemize}
    \item Se $n$ é par: \\
    Sejam $n_1, n_2 \in \N^*$ e ambos pares. Temos:
    \begin{align*}
        f(n_1) = f(n_2) &\implies \frac{-n_1}{2} = \frac{-n_2}{2} \\ &\implies
        (-n_1) = (-n_2) \\ &\implies
        n_1 = n_2
    \end{align*}
    \item Se $n$ é ímpar: \\
    Sejam $n_1, n_2 \in \N^*$ e ambos ímpares. Temos:
    \begin{align*}
        f(n_1) = f(n_2) &\implies \frac{n_1 - 1}{2} = \frac{n_2 - 1}{2} \\ &\implies
        n_1 - 1 = n_2 - 1 \\ &\implies 
        n_1 = n_2
    \end{align*}
    \item Se um elemento é ímpar e outro par: \\
    Sejam $n_1, n_2 \in \N^*$, onde $n_1$ é ímpar e $n_2$ é par. Temos:
    \begin{align*}
        n_1 > 0 & \implies -n_1 < 0 \\ & \implies
        \frac{-n}{2} < 0 \\ & \implies
        f(n_1) < 0
    \end{align*}
    Temos também que:
    \begin{align*}
        n_2 \ge 1 & \implies n_2 - 1 \ge 0 \\ & \implies
        \frac{n_2 - 1}{2} \ge 0 \\ & \implies
        f(n_2) \ge 0
    \end{align*}
    Dessa forma $n_1 \ne n_2 \implies f(n_1) \ne f(n_2)$.
\end{itemize}
Assim, está provada a injetividade da função $f$. Vejamos a sobrejetividade:
\begin{itemize}
    \item Se $n$ é par: \\
    Note que quando executada sobre os números pares a função $f$ gera um inteiro negativo. \\
    Seja $y \in \Z_-^*$. Note que:
    \begin{align*}
        y < 0 & \implies 2y < 0 \\ & \implies
        -2y > 0
    \end{align*}
    Logo, existe um número $-2y \in N^*$ tal que
    \begin{align*}
        f(-2y) = \frac{-(-2y)}{2} = y
    \end{align*}
    \item Se $n$ é ímpar: \\
    Note que executando a função $f$ sobre os números impares naturais, temos somente números inteiros não-negativos. \\
    Seja $y \in \Z_+$. Note que:
    \begin{align*}
        y \ge 0 & \implies 2y \ge 0 \\ & \implies
        2y + 1 \ge 1
    \end{align*}
    Logo, existe um número $2y + 1 \in \N^*$ tal que
    \begin{align*}
        f(2y + 1) = \frac{2y + 1 - 1}{2} = y
    \end{align*}
\end{itemize}
Portanto, está provada a sobrejetividade para ambos os casos de $n$. Consequentemente também está provada a bijetividade de $f$.