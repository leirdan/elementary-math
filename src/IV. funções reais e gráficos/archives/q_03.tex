\enquote{Considere a função $f: \N^* \rightarrow \Z$ tal que
\begin{align*}
f(n) = \left\{
        \begin{aligned}
            \frac{-n}{2} &; \quad \text{se $n$ é par} \\
            \frac{n - 1}{2} &; \quad \text{se $n$ é ímpar}
        \end{aligned}
        \right.
\end{align*}
Mostre que $f$ é bijetiva.} \\
\emph{Resolução}: Devemos mostrar que $f$ é injetiva e sobrejetiva para todos os casos. Começamos com a injetividade para:
\begin{itemize}
    \item Se $n$ é par: \\
    Sejam $n_1, n_2 \in \N^*$ e ambos pares. Temos:
    \begin{align*}
        f(n_1) = f(n_2) &\implies \frac{-n_1}{2} = \frac{-n_2}{2} \quad &(\cdot 2) \\ &\implies
        (-n_1) = (-n_2) \quad &(\cdot-1) \\ &\implies
        n_1 = n_2
    \end{align*}
    \item Se $n$ é ímpar: \\
    Sejam $n_3, n_4 \in \N^*$ e ambos ímpares. Temos:
    \begin{align*}
        f(n_3) = f(n_4) &\implies \frac{n_3 - 1}{2} = \frac{n_4 - 1}{2} \quad &(\cdot 2) \\ &\implies
        n_3 - 1 = n_4 - 1 \quad &(+1) \\ &\implies 
        n_3 = n_4
    \end{align*}
    \item Se um elemento é ímpar e outro par: \\
    Sejam $n_5, n_6 \in \N^*$, onde $n_5$ é ímpar e $n_6$ é par. Temos:
    \begin{align*}
        n_5 > 0 & \implies -n_5 < 0 \\ & \implies
        \frac{-n}{5} < 0 \\ & \implies
        f(n_5) < 0
    \end{align*}
    Temos também que:
    \begin{align*}
        n_6 \ge 1 & \implies n_6 - 1 \ge 0 \\ & \implies
        \frac{n_6 - 1}{2} \ge 0 \\ & \implies
        f(n_6) \ge 0
    \end{align*}
    Dessa forma, temos que $n_5 \ne n_6 \implies f(n_5) \ne f(n_6)$. Logo, a função é injetiva para todos os casos.
\end{itemize}
Assim, está provada a injetividade da função $f$. Vejamos a sobrejetividade:
\begin{itemize}
    \item Se $n$ é par: \\
    Note que, quando executada sobre os números pares, a função $f$ gera um inteiro negativo. \\
    Logo, seja $y \in \N^*, (-2y) \in \Z_-$, tal que $y$ é par. Portanto, ao aplicar a função sobre $(-2y)$:
    \begin{displaymath}
        f(-2y) = \frac{-(-2y)}{2} = y
    \end{displaymath}
    \item Se $n$ é ímpar: \\
    Note que executando a função $f$ sobre os números impares naturais, temos somente números inteiros não-negativos. \\
    Logo, seja $z \in \N^*$ e $(2z + 1) \in \Z_+$, tal que $z$ é ímpar. Assim, temos:
    \begin{displaymath}
        f(2z + 1) = \frac{(2z + 1) - 1}{2} = z
    \end{displaymath}
\end{itemize}
Portanto, está provada a sobrejetividade para ambos os casos de $n$. Consequentemente também está provada a bijetividade de $f$.