\emph{Resolução}: Tomando $n$ como uma função que calcula a quantidade de elementos de um conjunto, seja a função
\begin{align*}
    h: \mathbb{N} &\rightarrow \mathbb{N} \\
    X &\rightarrow n(X)
\end{align*}
\par Injetividade: \textbf{não}. Tomando $A = \{ 1, 2, 3\}$ e $B = \{6, 7, 8\}$, temos 
    \begin{displaymath}
    h(A) = 3 = h(B)
    \end{displaymath}
\par Sobrejetividade: \textbf{sim}. Seja $n \in \mathbb{N}$; temos um conjunto $X \subseteq \mathbb{N}$:
    \begin{displaymath}
        h(X) \implies n(X) = n
    \end{displaymath}
Portanto é sobrejetiva.