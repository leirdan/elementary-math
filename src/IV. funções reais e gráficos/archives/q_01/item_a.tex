\emph{Resolução}. Seja a função
\begin{align*}
    d: \mathbb{R}^2 &\rightarrow \mathbb{R}_+ \\
    (x, y) &\rightarrow \sqrt{(x - 0) + (y - 0)}
\end{align*}
\par Injetividade: \textbf{não}. Sejam os pontos $(6, 4), (4,6) \in \mathbb{R}$ e note que:
    \begin{align*}
        f(6,4) = \sqrt{10} = f(4,6)
    \end{align*}
\par Sobrejetividade: \textbf{sim}. Tomando $\sqrt{x + y} \in \mathbb{R}_+$ e $x, y \in \mathbb{R}$, temos:
    \begin{align*}
        f(x, y) = \sqrt{(x - 0) + (y - 0)} = \sqrt{x + y}
    \end{align*}