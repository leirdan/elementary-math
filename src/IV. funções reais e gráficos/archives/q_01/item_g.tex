\emph{Resolução}: Tomando $inf$ como uma função que calcula o menor elemento de um conjunto, seja a função
\begin{align*}
    i: \mathbb{N}\backslash \emptyset &\rightarrow \mathbb{N} \\
    X &\rightarrow inf(X)
\end{align*}
\par Injetividade: \textbf{não}. Sendo os conjuntos $A = \{ 1, 3, 5, 7\}$ e $B = \{ 1, 2, 4, 8 \}$, note que
    \begin{displaymath}
        i(A) = 1 = i(B)
    \end{displaymath}
\par Sobrejetividade: \textbf{sim}. Tome $C \subseteq \mathbb{N}$ e $n \in C$; então, temos:
    \begin{displaymath}
        i(C) \implies inf(C) = n
    \end{displaymath}
    Portanto é sobrejetiva.