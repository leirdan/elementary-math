\emph{Resolução}: Seja a função
\begin{align*}
    c: \mathbb{R} &\rightarrow \mathbb{R} \\
    R &\rightarrow R^C
\end{align*}
\par Injetividade: \textbf{sim}. Sejam $A, B \subseteq \mathbb{R}$, de modo que $A \ne B$. Então temos:
    \begin{displaymath}
        A \ne B \implies c(A) \ne c(B) \implies A^C \ne B^C
    \end{displaymath}
    Portanto é injetiva.
\par Sobrejetiva: \textbf{sim}. Seja $N^C \subseteq \mathbb{R}$; logo, existe $N \subseteq \mathbb{R}$ de modo que
    \begin{displaymath}
        c(N) = N^C
    \end{displaymath}
    Portanto é sobrejetiva e, além disso, é também bijetiva.