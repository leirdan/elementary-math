\enquote{Considere a função $g: [0; 5] \rightarrow \mathbb{R}$ definida por:}
        \begin{align*}
            g(x) = \left\{
                \begin{aligned}
                    4x - x^2 &; \quad x < 3 \\
                    x - 2 &; \quad x \ge 3
                \end{aligned}
                \right.
        \end{align*}
        \enquote{Determine as soluções de:}
        \begin{enumerate}
            \item $g(x) = -1$;
            \begin{enumerate}
                \item Se $x < 3$: \\ 
                    Temos $4x - x^2 = -1 \implies 4x - x^2 + 1 = 0$. Aplicando lei de Bhaskara, temos:
                    \begin{align*}
                        x &= \frac{-4 \pm \sqrt{4^2 - (-4)}}{-2} \\ \implies 
                        x &= \frac{-4 \pm \sqrt{20}}{-2} \\ \implies
                        x &= \frac{-4 \pm 2\sqrt{5}}{-2}
                    \end{align*}
                    Portanto, tomando $x_1, x_2 \in \mathbb{R}$, temos as possibilidades:
                    \begin{align*}
                        &x_1 = \frac{-4 + 2\sqrt{5}}{-2} \implies 2 - \sqrt{5} \\ 
                        &x_2 = \frac{-4 -2\sqrt{5}}{-2} \implies 2 + \sqrt{5}   
                    \end{align*}
                    Como $x_2 < 3$ e $x_1 \notin [0; 5]$, temos que não há conjunto solução para este caso. 
                \item Se $x \ge 3$: \\
                    Temos \begin{displaymath}
                        x - 2 = -1 \implies x - 1 = 0 \implies x = 1.
                    \end{displaymath}
                    Temos que $x$ não é maior ou igual a três. Logo, não há solução para quando $g(x) = -1$.
                \end{enumerate}
            \item $g(x) = 0$;
            \begin{enumerate}
                \item Se $x < 3$: \\
                    Temos:
                    \begin{displaymath}
                        4x - x^2 = 0
                    \end{displaymath}
                    Aplicando lei de Bhaskara, temos:
                    \begin{align*}
                        x &= \frac{(-4) \pm \sqrt{16 - 0}}{-2} \\ \implies
                        x &= \frac{(-4) \pm 4}{-2}
                    \end{align*}
                    Tomando $x_1, x_2 \in \mathbb{R}$, temos as seguintes possibilidades:
                    \begin{align*}
                        &x_1 = \frac{(-4) - 4}{-2} \implies x_1 = 4 \\  
                        &x_2 = \frac{(-4) + 4}{-2} \implies x_2 = 0
                    \end{align*}
                    Temos que $x_1 > 3$, logo não é solução. Então, seja $S_1$ o seguinte conjunto solução:
                    \begin{align*}
                        S_1 &= \{0\} \cap ]-\infty; 3[ \cap  [0; 5] = 0
                    \end{align*}
                \item Se $x \ge 3$: \\
                    Temos: 
                    \begin{displaymath}
                        x - 2 = 0 \implies x = 2
                    \end{displaymath}
                    Temos que $x$ não é maior ou igual a 3. Portanto a solução para $g(x) = 0$ é somente 0.
            \end{enumerate}
        \item $g(x) = 3$;
            \begin{enumerate}
            \item Se $x < 3$: \\
                Temos:
                \begin{displaymath}
                    4x - x^2 = 3 \implies 4x - x^2 - 3 = 0 
                \end{displaymath}
                Aplicando lei de Bhaskara, temos:
                \begin{align*}
                    x &= \frac{-4 \pm \sqrt{16 - (-12)}}{-2} \\ \implies
                    x &= \frac{-4 \pm 2\sqrt{7}}{-2}
                \end{align*}
                Tomemos as possibilidades $x_1, x_2 \in \mathbb{R}$:
                \begin{align*}
                    x_1 &= \frac{-4 + 2\sqrt{7}}{-2} \implies x_1 = 2 - \sqrt{7} ; \\ 
                    x_2 &= \frac{-4 - 2\sqrt{7}}{-2} \implies x_2 = 2 + \sqrt{7}
                \end{align*}
                Temos que $x_1 \notin [0;5]$ e $x_2$ não é menor que 3. Logo, não há solução para este caso.
            \item Se $x \ge 3$: \\
                Temos:
                \begin{displaymath}
                    x - 2 = 3 \implies x = 5
                \end{displaymath}
                Seja $S$, então, o seguinte conjunto solução:
                \begin{displaymath}
                    S = 5 \cap [3; \infty[ \cap [0; 5] = 5
                \end{displaymath}
                Portanto, a única solução para quando $g(x) = 3$ é 5.
            \end{enumerate}
        \item Se $g(x) = 4$:
        \begin{enumerate}
            \item Caso $x < 3$: \\
                Temos:
                \begin{displaymath}
                    4x - x^2 = 4 \implies 4x - x^2 - 4 = 0
                \end{displaymath}
                Aplicando lei de Bhaskara, temos:
                \begin{align*}
                    x &= \frac{-4 \pm \sqrt{16 - 16}}{-2} \\ \implies
                    x &= \frac{-4 \pm 0}{-2} \\ \implies 
                    x &= 2
                \end{align*}
                Temos o conjunto solução $S_1$ tal que:
                \begin{displaymath}
                    S_1 = 2 \cap ]-\infty; 3[ \cap [0; 5] = 2
                \end{displaymath}
            \item Caso $x \ge 3$: \\
                Temos:
                \begin{displaymath}
                    x - 2 = 4 \implies x = 6
                \end{displaymath}
                Temos $S_2$ como o seguinte conjunto solução:
                \begin{displaymath}
                    S_2 = 6 \cap [3; \infty[ \cap [0; 5] = \emptyset
                \end{displaymath}
                Portanto, temos que a única solução para $g(x) = 4$ é 2.
            \end{enumerate}
        \item Se $g(x) < 3$:
        \begin{enumerate}
            \item Caso $x < 3$: \\
            \textbf{Em desenvolvimento.}
            % Temos:
            % \begin{displaymath}
            %     4x - x^2 < 3 \implies (x - 4)\cdot(-x + 0) < 3
            % \end{displaymath}
            % Conforme estudo do sinal temos que o conjunto-solução para este caso é: CORRIGIRRRRR
            % \begin{displaymath}
            %     S_1 = ]0; 7[ \cap ]-\infty; 3[ \cap [0; 5] = ]0; 3[
            % \end{displaymath}
            \item Caso $x \ge 3$: \\
            Temos: 
            \begin{displaymath}
                x - 2 \ge 3 \implies x \ge 5
            \end{displaymath}
            Temos o seguinte conjunto-solução para este caso:
            \begin{displaymath}
                S_2 = [5; \infty[ \cap [3; \infty[ \cap [0; 5] = 5
            \end{displaymath}
            Temos, portanto, que as soluções para $g(x) < 3$ são
            \begin{displaymath}
                \text{\textbf{Em desenvolvimento.}}
            \end{displaymath}
        \end{enumerate}
    \end{enumerate}