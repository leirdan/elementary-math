\enquote{Considere 
\begin{displaymath}
    f: [3,5; +\infty[ \rightarrow  [-2,25; + \infty[
\end{displaymath}
tal que $f(x) = x^2 - 7x + 10$. Prove que $f$ é bijetiva.}
\\ \emph{Resolução}. Provemos primeiro sua sobrejetividade. Seja $y \in [-2,25; + \infty[$. Note que: 
\begin{align*}
    y \ge -\frac{9}{4} & \implies y + \frac{9}{4} \ge 0 \\ & \implies
    \sqrt{y + \frac{9}{4}} \ge 0 \\ & \implies
    \sqrt{y + \frac{9}{4}} + \frac{7}{2} \ge \frac{7}{2}
\end{align*}
Logo, existe um número $\sqrt{y + \frac{9}{4}} + \frac{7}{2}$ tal que
\begin{align*}
    f(\sqrt{y + \frac{9}{4}} + \frac{7}{2}) & = (\sqrt{y + \frac{9}{4}} + \frac{7}{2})^2 - 7 \cdot (\sqrt{y + \frac{9}{4}} + \frac{7}{2}) + 10 \\ & =
    y + \frac{9}{4} + 2\cdot\frac{7}{2}\cdot \sqrt{y + \frac{9}{4}} + \frac{49}{4} -7\sqrt{y + \frac{9}{4}} - \frac{49}{2} + 10 \\ & =
    y + \frac{9}{4} + \frac{49}{4} - \frac{49}{2} + 10 \\ & =
    y + \frac{58}{4} - \frac{29}{2} \\ & =
    y + \frac{58 - 58}{4} \\ & =
    y
\end{align*}
Portanto, está provada a sobrejetividade. \\
Por fim, provemos sua injetividade. Tomando $a, b \in [3,5; +\infty[$, temos que
\begin{align*}
    f(a) = f(b) & \implies a^2 - 7a + 10 = b^2 - 7b + 10 \\ & \implies
    a^2 - 7a = b^2 - 7b \\ & \implies
    a^2 - 7a + (\frac{7}{2})^2 = b^2 - 7a + (\frac{7}{2})^2 \\ & \implies
    (a - \frac{7}{2})^2 = (b - \frac{7}{2})^2 \\ & \implies
    \sqrt{(a - \frac{7}{2})^2} = \sqrt{(b - \frac{7}{2})^2} \\ & \implies
    |a - \frac{7}{2}| = |b - \frac{7}{2}| \\ & \implies
    a - \frac{7}{2} = b - \frac{7}{2} \quad (a - \frac{7}{2} \ge 0, b - \frac{7}{2} \ge 0) \\ & \implies
    a = b
\end{align*}
Assim está provada a injetividade e, consequentemente, a bijetividade da função.