\enquote{Seja a função $f: [3; 5] \rightarrow \R$ tal que $f(x) = -x^2 + 4x - 3$.}
\begin{enumerate}
    \item \enquote{Mostre que $f$ é decrescente.} \\
        \emph{Resolução}. Sejam $a, b \in [3;5]$ tal que $a > b$. Temos, portanto, para todos $a, b$:
        \begin{align*}
            3 \le a > b & \implies
            1 \le a - 2 > b - 2 \\ & \implies
            (a - 2)^2 > (b - 2)^2 \\ & \implies
            a^2 - 4a + 4 > b^2 - 4b + 4 \quad & \cdot(-1)\\ & \implies
            -a^2 + 4a - 4 < -b^2 + 4b - 4 \\ & \implies 
            -a^2 + 4a - 3 < -b^2 + 4b - 3 \quad & +1 \\ & \implies
            f(a) < f(b)
        \end{align*} 
        Logo, está mostrada que a função é decrescente.
    \item \enquote{$f$ possui máximo absoluto? Se sim, ocorre em qual ponto?} \\
    \emph{Resolução.} Seja $x \in [3;5]$. Então, temos
    \begin{displaymath}
        3 \le x \le 5
    \end{displaymath} 
    Como $f$ é decrescente, temos
    \begin{align*}
        f(3) \ge f(x),
    \end{align*}
    para todo $x \in [3;5]$. Ou seja, $x_0 = 3$ é o máximo absoluto de $f$.
    \item \enquote{$f$ possui mínimo absoluto? Se sim, ocorre em qual ponto?} \\
    \emph{Resolução.} Seja $x \in [3;5]$. Então temos
    \begin{displaymath}
        3 \le x \le 5
    \end{displaymath} 
    Como $f$ é decrescente, temos
    \begin{align*}
        f(5) \le f(x),
    \end{align*}
    para todo $x \in [3;5]$. Ou seja, $x_0 = 5$ é o mínimo absoluto de $f$.
\end{enumerate}