\enquote{Seja a função $f: [3; 5] \rightarrow \R$ tal que $f(x) = -x^2 + 4x - 3$.}
\begin{enumerate}
    \item \enquote{Mostre que $f$ é decrescente.} \\
        \emph{Resolução}. Sejam $a, b \in [3;5]$ tal que $a > b$. Temos, portanto, para todos $a, b$:
        \begin{align*}
            a > b & \implies
            a - 2 > b - 2 \\ & \implies
            (a - 2)^2 > (b - 2)^2 \\ & \implies
            a^2 - 4a + 4 > b^2 - 4b + 4 \quad & \cdot(-1)\\ & \implies
            -a^2 + 4a - 4 < -b^2 + 4b - 4 \\ & \implies 
            -a^2 + 4a - 3 < -b^2 + 4b - 3 \quad & +1 \\ & \implies
            f(a) < f(b)
        \end{align*} 
        Logo, está mostrada que a função é decrescente.
    \item \enquote{$f$ possui máximo absoluto? Se sim, ocorre em qual ponto?} \\
    \textbf{Aguardando validação.}
    % \emph{Resolução}. Seja $x_0, x \in [3; 5]$. Para qualquer $x$, temos:
    % \begin{align*}
    %     x_0 \ge x & \implies x_0 - 2 \ge x - 2 \\ & \implies
    %     (x_0 - 2)^2 \ge (x - 2)^2 \\ & \implies
    % \end{align*}
    \item \enquote{$f$ possui mínimo absoluto? Se sim, ocorre em qual ponto?} \\
    \textbf{Aguardando validação.}
    %     \emph{Resolução.} Seja $x_0, x \in [3; 5]$. Para qualquer $x$, temos:
    %     \begin{align*}
    %         x_0 \le x & \implies x_0 - 2 \le x - 2 \\ & \implies
    %         (x_0 - 2)^2 \le (x - 2)^2 \\ & \implies 
    %         x_0^2 - 4x_0 + 4 \le x^2 - 4x + 4 \\ & \implies
    %         -x_0^2 + 4x_0 - 4 \ge -x^2 + 4x - 4 \\ & \implies
    %         -x_0^2 + 4x_0 - 3 \ge -x^2 + 4x - 3 \\ & \implies
    %         f(x_0) \ge f(x)
    %     \end{align*}
    % Logo, a função $f$ não tem mínimo absoluto.
\end{enumerate}