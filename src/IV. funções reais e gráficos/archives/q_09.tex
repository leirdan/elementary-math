\enquote{Considere as funções reais $f: X \rightarrow Y, g: Y \rightarrow Z$. Demonstre ou refute com contraexemplos as afirmações abaixo:}
    \begin{enumerate}
    \item \enquote{Se $f$ e $g$ são injetivas, então ($g \circ f$) é injetiva;} 
    \\ \emph{Resolução.} Temos que ($g \circ f$) é definida tal que
    \begin{displaymath}
        g \circ f: X \rightarrow Z 
    \end{displaymath}
    Se $f$ é injetiva e $g$ também, tomando $a, b \in X \text{ e } x, y \in Y$ temos:
    \begin{align*}
        (g \circ f)(a) = (g \circ f)(b) & \implies
        g(f(a)) = g(f(b)) \\ & \implies
        g(x) = g(y) \\ & \implies
        x = y
    \end{align*}
    Portanto, ($g \circ f$) é de fato injetiva e a afirmação é verdadeira.
    \item \enquote{Se $(g \circ f)$ é injetiva, então $f$ e $g$ são injetivas.} 
    \\ \emph{Resolução.} Sejam as funções $f: \{x_1\} \rightarrow \{y_1, y_2\}, g: \{y_1, y_2\} \rightarrow \{z_1\}, (g \circ f): \{x_1\} \rightarrow \{z_1\}$ tal que $f(x_1) = y_1, g(y_1) = z_1, g(y_2) = z_1$.
    Note que $(g \circ f)$ é injetiva pois $(g \circ f)(x_1) = (z_1)$, mas $g$ não é pois $g(y_1) = z_1 = g(y_2)$.
    Logo, a afirmação é falsa.
    \item \enquote{Se $f$ e $g$ são sobrejetivas, então $(g\circ f)$ é sobrejetiva.}
    \\ \emph{Resolução}. Temos que ($g \circ f$) é definida tal que
    \begin{displaymath}
        g \circ f: X \rightarrow Z 
    \end{displaymath}
    Desse modo, se $f$ e $g$ são sobrejetivas, e tomando $x \in X, y \in Y$ e $z \in Z$, temos
    \begin{displaymath}
        (g\circ f)(x) = g(f(x)) = g(y) = z
    \end{displaymath}
    Logo, $(g \circ f)$ é sobrejetiva e a afirmação é verdadeira.
    \item \enquote{Se $(g \circ f)$ é sobrejetiva, então $f \text{ e } g$ são sobrejetivas.}
    \\ \emph{Resolução.} Sejam as funções
    \begin{align*}
        f: \mathbb{R} &\rightarrow \mathbb{R}^*_+ \\
        x &\rightarrow x^2
    \end{align*}
    \begin{align*}
        g: \mathbb{R}^*_+ &\rightarrow \mathbb{R}_+ \\
        y &\rightarrow \sqrt{y},
    \end{align*}
    \begin{align*}
        g \circ f: \mathbb{R} &\rightarrow \mathbb{R}_+ \\
        x \rightarrow &\sqrt{x^2} \rightarrow y
    \end{align*}
    Tomando $y \in \mathbb{R}_+$, temos que existe $x \in \mathbb{R}$ tal que 
    \begin{displaymath}
        g(f(x)) = g(x^2) = y 
    \end{displaymath}
    Logo, a função $(g \circ f)$ é sobrejetiva. Note, contudo, que $g$ não é sobrejetiva pois, tomando $0 \in \mathbb{R}_+$, não há nenhum elemento de $\mathbb{R}^*_+$ que resulte nele.
\end{enumerate}