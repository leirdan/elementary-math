\enquote{Faça uso de pelo menos um dos resultados anteriores para mostrar a injetividade das funções:}
\begin{displaymath}
    f: [1; \infty[ \rightarrow ]-\infty; 0], \text{ tal que } f(x) = -x + 1, 
\end{displaymath}
\begin{displaymath}
    g: [1; \infty[ \rightarrow \R, \text{ tal que } g(x) = x^2 - 2x - 3,
\end{displaymath}
\begin{displaymath}
    h: ]-\infty; 0] \rightarrow \R, \text { tal que } h(x) = x^2 - 4
\end{displaymath} \\
\emph{Resolução.}
\begin{enumerate}
    \item Função $f$; \\
    Sejam $x_1, x_2 \in [1; \infty[$ tal que
    \begin{align*}
        f(x_1) = f(x_2) & \implies -x_1 + 1 = -x_2 + 1 \\ & \implies
        x_1 = x_2
    \end{align*}
    \item Função $h$; \\
    Sejam $x_1, x_2 \in ]-\infty; 0]$ tal que
    \begin{align*}
        h(x_1) = h(x_2) & \implies x_1^2 - 4 = x_2^2 - 4 \\ & \implies
        \sqrt{x_1^2} = \sqrt{x_2^2} \\ & \implies
        |x_1| = |x_2| \\ & \implies
        x_1 = x_2
    \end{align*}
    \item Função $g$; \\
    Se $f$ e $h$ são injetivas, então $(h \circ f): [1; \infty[ \rightarrow \R$ também é injetiva. Como $g: [1; \infty[ \rightarrow \R$ e para qualquer $x \in [1; \infty[$ temos que
    \begin{align*}
        (h \circ f)(x) & = h(f(x)) \\ & =
        h(-x + 1) \\ & =
        (-x + 1)^2 - 4 \\ & =
        x^2 - 2x + 1 - 4 \\ & =
        x^2 - 2x - 3 \\ & =
        g(x)         
    \end{align*}
    A função $g$ também é, portanto, injetiva pois $(h \circ f) = g$.
\end{enumerate}