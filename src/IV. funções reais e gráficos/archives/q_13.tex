\enquote{Considere a função $f: \R_- \rightarrow \R_+^*$ tal que $f(x) = \frac{1}{1 + x^2}$. Responda as seguintes perguntas apresentando as respectivas justificativas.}
\begin{enumerate}
    \item \enquote{$f$ é monótona? Se sim, de que tipo? Se não, $f$ possui algum intervalo de monotonicidade?} \\
    \emph{Resolução.} Sim, é monótona crescente pois, tomados quaisquer $a, b \in \R_-$, temos
    \begin{align*}
        a < b \le 0 & \implies a^2 > b^2 \ge 0 \\ & \implies
        1 + a^2 > 1 + b^2 > 1 \ge 0 \\ & \implies
        \frac{1}{1 + a^2} < \frac{1}{1 + b^2} \\ & \implies
        f(a) < f(b)
    \end{align*}
    A função é, portanto, crescente.
    \item \enquote{$f$ possui máximo absoluto?} \\
    \emph{Resolução.} Seja $x \in \R_-$, ou seja, $x \le 0$. Logo, como a função é crescente, temos que para quaisquer $x$:
    \begin{align*}
        f(x) \le f(0)
    \end{align*}
    Então a função possui um máximo absoluto $x_0$ tal que $x_0 = 0$. 
    \item \enquote{$f$ possui mínimo absoluto?} \\
    \emph{Resolução.} Sejam $x_0, x \in \R_-$. Suponha, por absurdo, que $x$ é o ponto mínimo de $f$; então, $f(x_0) \le f(x)$ para todo $x \in \R_-$. \\ Note, contudo, que temos $x_1 \in \R_-$ tal que $x_1 < x_0$. Desse modo, como a função é crescente, temos que
    \begin{displaymath}
        f(x_1) \le f(x_0),
    \end{displaymath}
    o que contradiz $x_0$ ser o mínimo absoluto. Logo, a função não tem mínimo absoluto.
    \item \enquote{$f$ é limitada?} \\
    \emph{Resolução.} Sim. $f$ é limitada superiormente pelo fato de ter um máximo absoluto; e também é limitada inferiormente pois, para qualquer $x$, temos $f(x) \ge 0$.
\end{enumerate}