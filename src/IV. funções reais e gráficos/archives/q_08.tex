\enquote{Considere as funções:}
\begin{displaymath}
    f: ]-\infty; 0] \rightarrow [-4; \infty[, \text{ tal que } f(x) = -x - 4, 
\end{displaymath}
\begin{displaymath}
    g: ]-\infty; 0] \rightarrow \R, \text{ tal que } g(x) = \sqrt{-x}
\end{displaymath}
\begin{displaymath}
    h: \R \rightarrow [-4; \infty[, \text{ tal que } h(x) = x^2 - 4
\end{displaymath}
\enquote{Quais dessas funções é sobrejetiva e quais não são? Alguma dessas funções é resultante da composição das outras?} \\
\emph{Resolução.} Note que $f$ é sobrejetiva pois, tomando $y \in [-4; \infty[$, temos
\begin{align*}
    y \ge -4 \implies -y \le 4 \implies -4 -y \le 0 
\end{align*}
Logo, há um número $-4 - y \in ]-\infty; 0]$ tal que 
\begin{align*}
    f(-4 - y) = -(-4 - y) - 4 = y 
\end{align*}
Além disso, $f$ é a composta de $(h \circ g)$, pois tomado $x \in \R_-$ temos
\begin{align*}
    h(g(x))  = h(\sqrt{-x}) = (\sqrt{-x})^2 - 4 = -x - 4 = f(x)    
\end{align*}
$h$ também é sobrejetiva pois, tomando $y \in [-4; \infty[$, temos que há um número $\sqrt{y + 4} \in \R$ tal que
\begin{align*}
    h(\sqrt{y + 4}) = (\sqrt{y + 4})^2 - 4 = y
\end{align*}
Contudo, $g$ não é sobrejetiva pois $g(x) \ge 0$ para todos os $x \in \R_-$; logo, não existe $g(x) = -1$.