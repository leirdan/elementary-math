\enquote{Considere a função $f: \R^* \rightarrow \R^*_+$ tal que $f(x) = \dfrac{1}{1 + x^2}$. Responda as seguintes perguntas apresentando as respectivas justificativas:}
\begin{enumerate}
    \item \enquote{$f$ é injetiva}? \\
    \textbf{Não}. Note que, tomando $1, -1 \in \R^*$, temos
    \begin{displaymath}
        f(-1) = \frac{1}{2} = f(1)
    \end{displaymath}
    Logo, a função $f$ não é injetiva.
    \item \enquote{$f$ é sobrejetiva}? \\
    \textbf{Não}. Note que, sendo $1 \in \R^*_+$:
    \begin{align*}
        f(x) = 1 & \implies \frac{1}{1 + x^2} = 1 \\ & \implies
        1 = 1 \cdot (1 + x^2) \\ & \implies
        1 = 1 + x^2 \\ & \implies
        x^2 = 0
    \end{align*}
    Como $0 \notin \R^*$, temos que a função não é sobrejetiva, pois não há $x \in \R^*$ tal que $f(x) = 1$.
\end{enumerate}