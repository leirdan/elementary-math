\enquote{Um explorador descobriu, na selva amazônica, uma espécie nova de planta e, pesquisando-a durante anos, comprovou que seu crescimento médio variava de acordo com a fórmula $A = 40 \cdot 1,1^t$, onde a altura média $A$ é medida em centímetros e o tempo $t$ em anos. Sabendo-se que $\log 2 \approx 0,30$ e $\log 11 \approx 1,04$, determine:}
\begin{enumerate}
    \item A altura média, em centímetros, de uma planta dessa espécie aos 3 anos de vida; \\
    \emph{Resolução.} Tomando $t = 3$, temos:
    \begin{align*}
        A & = 40 \cdot 1,1^3 \\ & =
        40 \cdot 1,331 \\ & =
        53,24 
    \end{align*}
    Logo, a planta tem altura média de 53,24cm em 3 anos.
    \item A idade, em anos, na qual a planta tem uma altura média de $1,6m$. \\
    \emph{Resolução.} Tomando $A = 160$, temos:
    \begin{align*}
        160 = 40 \cdot 1,1^t & \implies 4 = 1,1^t \\ & \implies t = \log_{1,1} 4 \\ & \implies
        t = \frac{\log 4}{\log 1,1} \\ & \implies
        t = \frac{2 \cdot \log 2}{\log 11 + \log \frac{1}{10}} \\ & \implies
        t = \frac{0,6}{1,04 - 1} \quad & (\log \frac{1}{10} = -1)\\ & \implies
        t = 1,5
    \end{align*}
    Logo, a planta terá 1 ano e 5 meses quando tiver uma altura de 1,6m.
\end{enumerate}