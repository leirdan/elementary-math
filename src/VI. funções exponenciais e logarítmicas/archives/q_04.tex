\enquote{Use as aproximações $\log 2 \approx 0,301, \log 3 \approx 0,477, \log 5 \approx 0,699$ para obter valores aproximados para:}
\begin{enumerate}
    \item $\log 9$: pela propriedade (a) temos:
    \begin{align*}
        \log(9) & = \log(3 \cdot 3) \\ & = 
        \log 3 + \log 3 \\ & = 0,477 + 0,477 \\ & = 0,954
    \end{align*}
    \item $\log 40$: pelas propriedades (a) e (b) temos:
    \begin{align*}
        \log 40 & = \log (5 \cdot 2^3) \\ & = 
        \log 5 + \log 2^3 \\ & = 
        \log 5 + 3 \cdot \log 2 \\ &= 
        0,699 + 0,903 \\ & =
        1,602
    \end{align*}
    \item $\log 200$: pelas propriedades (a) e (b) temos:
    \begin{align*}
        \log 200 & = \log (5^2 \cdot 2^3) \\ & =
        2 \cdot \log 5 + 3 \cdot \log 2 \\ & =
        1,398 + 0,903 \\ & = 
        2,301
    \end{align*}
    \item $\log 3000$: pelas propriedades (a) e (b) temos:
    \begin{align*}
        \log 3000 & = \log (200 \cdot 15) \\ & =
        \log 200 + \log 15 \\ & =
        \log 200 + \log (5 \cdot 3) \\ & =
        \log 200 + \log 5 + \log 3 \\ & =
        2,301 + 0,699 + 0,477 \\ & =
        3,477
    \end{align*}
    \item $\log 0,003$: pelas propriedades (a) e (b) temos:
    \begin{align*}
        \log 0,003 & = \log (3 \cdot 10^{-3}) \\ & =
        \log 3 + (-3 \cdot \log 10) \\ & =
        0,477 - 3 \cdot (\log 5 \cdot \log 2) \\ & =
        0,477 - 3 \cdot 0,699 + 0,301 \\ & =
        0,477 - 3 \cdot 1 \\ & =
        -2,523
    \end{align*}
    \item $\log 0,81$: pelas propriedades (a) e (b) temos:
    \begin{align*}
        \log 0,81 & = \log (9^2 \cdot 10^{-2}) \\ & =
        2 \cdot \log 9 -2 \cdot \log 10 \\ &=
        1,908 - 2 \\ & = -0,092
    \end{align*}
\end{enumerate}