\enquote{Decida quais das afirmações a seguir estão corretas. Justifique suas respostas.}
    \begin{enumerate}
        \item $\emptyset \in \emptyset$;
        \begin{itemize}
            \item \textbf{Proposição falsa}: se $\emptyset \in \emptyset$, significa que $\emptyset$ tem ao menos um elemento, o que vai contra sua definição e nos leva a um absurdo.
        \end{itemize}
        \item $\emptyset \subseteq \emptyset$;
        \begin{itemize}
            \item \textbf{Proposição verdadeira}: suponha que $\emptyset \nsubseteq \emptyset$. Logo, há um elemento $x \in \emptyset$ que não pertence a $\emptyset$. Isso gera um absurdo pois, por definição, o conjunto vazio não contém elementos. Logo, a proposição é verdadeira.
        \end{itemize}
        \item $\emptyset \in \{\emptyset\}$;
        \begin{itemize}
            \item \textbf{Proposição verdadeira}: tomado o conjunto $A = \{\emptyset\}$, suponha que $\emptyset \notin \{\emptyset\}$. No entanto, sabendo que o conjunto $A$ tem $\emptyset$ como elemento, chegamos a uma contradição. Logo, a proposição é verdadeira.
        \end{itemize}
        \item $\emptyset \subseteq \{\emptyset\}$.
        \begin{itemize}
            \item \textbf{Proposição verdadeira}: tome o conjunto $A = \{\emptyset\}$. De acordo com a Inclusão Universal do $\emptyset$, para todo conjunto $A$, vale $\emptyset \subseteq A$. Desse modo, a proposição é verdadeira.
        \end{itemize}
    \end{enumerate}