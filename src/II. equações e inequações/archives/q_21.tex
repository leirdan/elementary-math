\enquote{Determine o conjunto solução de cada uma das equações ou inequações modulares abaixo:}
\begin{enumerate}
    \item $|3x - 5| = 7$;
    \\ \emph{Resolução}. Temos por definição que:
    \begin{displaymath}
        |3x - 5| = \left\{\begin{array}{cc}
             3x-5, \text{ se } 3x-5 \ge 0 \implies & x \ge \dfrac{5}{3} \\
             -(3x-5), \text{ se } 3x-5 < 0 \implies & x < \dfrac{5}{3}
        \end{array}
        \right.
    \end{displaymath}
    Analisemos cada um dos casos:
    \begin{itemize}
        \item Se $x \ge \dfrac{5}{3}$: \emph{Verdadeiro}. 4 é maior que $\dfrac{5}{3}$.
        \begin{align*}
            3x - 5  = 7 & \implies 3x = 12 \\ & \implies x = 4 
        \end{align*}
        \item Se $x < \dfrac{5}{3}$: \emph{Verdadeiro}. O número negativo é menor que $\dfrac{5}{3}$.
        \begin{align*}
            -(3x - 5) = 7 & \implies -3x + 5 = 7 \\ & \implies -3x = 2 \\ & \implies
            x = -\frac{2}{3}
        \end{align*}
    \end{itemize}
    Desse modo o conjunto solução será
    \begin{displaymath}
        S = \{ -\dfrac{2}{3}, 4 \}
    \end{displaymath}
    \item $|-x + 8| = -1$;
    \\ \emph{Resolução}. Temos por definição:
    \begin{displaymath}
        |-x + 8| = \left\{\begin{array}{cc}
             -x + 8, \text{ se } -x + 8 \ge 0 \implies x \le 8&  \\
             -(-x + 8), \text{ se } -x + 8 < 0 \implies x > 8& 
        \end{array} 
        \right.
    \end{displaymath}
    Analisemos cada um dos casos:
    \begin{itemize}
        \item Se $x \le 8$; \emph{Falso}. Chegamos em um absurdo, pois 9 não é menor ou igual a 8.
        \begin{align*}
            -x + 8 = -1 & \implies -x = -9 \\ & \implies x = 9
        \end{align*}
        \item Se $x > 8$; \emph{Falso}. Chegamos em um absurdo.
        \begin{align*}
            -(-x + 8) = -1 & \implies x - 8 = -1 \\ & \implies x = 7 
        \end{align*}
    \end{itemize}
    Logo não há conjunto solução para essa equação.
    \item $|x^2 - 9| = 7$
    \\ \emph{Resolução.} Note que $x^2 - 9 = (x - 3) \cdot (x + 3)$. Temos por definição:
    \begin{displaymath}
        |x^2 - 9| = \left\{\begin{array}{cc}
             x^2 - 9, \text{ se } x^2 - 9 \ge 0 \implies (x - 3) \cdot (x + 3) \ge 0 &  \\
             - (x^2 - 9), \text{ se } x^2 - 9 < 0 \implies (x - 3) \cdot (x + 3) < 0 & 
        \end{array} \right.
    \end{displaymath}
    Fazendo o estudo do sinal temos os seguintes conjunto-solução para os casos:
    \begin{itemize}
        \item Se $x \in ]-\infty; -3]$ ou $x \in [3; \infty[$:
        \begin{align*}
            |x^2 - 9| = 7 & \implies x^2 - 9 = 7 \\ & \implies
            x^2 = 16 \\ & \implies
            x = \pm 4
        \end{align*}
        Temos, portanto, que esse caso é verdadeir, pois $-4 \in ]-\infty; -3]$ e $4 \in [3; \infty[$.
        \item Se $x \in ]-3; 3[$:
        \begin{align*}
            |x^2 - 9| = 7 & \implies -x^2 + 9 = 7 \\ & \implies
            -x^2 = -2 \\ & \implies
            x^2 = 2 \\ & \implies
            x = \sqrt{2}
        \end{align*}
        Temos que este caso também é verdadeiro. Assim podemos concluir que o conjunto solução desta equação modular é:
        \begin{displaymath}
            S = \{-4; \sqrt{2}; 4\}
        \end{displaymath}
    \end{itemize}
    \item $|x^2 - 1| = 3$;
    \\ \emph{Não fiz}.
    \item $|x + 1| + |-3x + 2| = 6$;
    \\ \emph{Resolução.} Temos por definição:
    \begin{displaymath}
        |x + 1| = \left\{\begin{array}{cc}
             x + 1, \text{ se } x + 1 \ge 0 \implies x \ge -1&  \\
             -(x + 1), \text{ se } x + 1 < 0 \implies x < -1&  \\
        \end{array} \right.
        \end{displaymath}
        \begin{displaymath}
        |-3x + 2| = \left\{\begin{array}{cc}
             -3x + 2, \text{ se } -3x + 2 \ge 0 \implies x \ge \frac{2}{3} &  \\
             -(-3x + 2), \text{ se } -3x + 2 < 0\implies x < \frac{2}{3}& 
        \end{array} \right.
    \end{displaymath}
    Podemos resolver tal situação com partição de intervalos:
    \begin{itemize}
        \item Se $x < -1$: \emph{Falso}. Chegamos em um absurdo, então não há solução para este caso.
        \begin{align*}
            -x - 1 + 3x - 2 = 6 & \implies 2x - 9 = 0 \\ & \implies x = \frac{9}{2}
        \end{align*}
        \item Se $-1 \le x < \frac{2}{3}$: \emph{Falso}. Não há solução para este caso.
        \begin{align*}
            x + 1 + 3x - 2 = 6 & \implies 4x - 7 = 0 \\ & \implies x = \frac{7}{4}
        \end{align*}
        \item Se $x \ge \frac{2}{3}$: \emph{Falso}. Não há solução para este caso.
        \begin{align*}
            x + 1 -3x + 2 = 6 & \implies -2x = 3 \\ & \implies x = -\frac{3}{2}
        \end{align*}
    \end{itemize}
    Portanto temos que esta equação modular não tem soluções.
    \item $|x - 1| \cdot |x + 2|  = 3$;
    \\ \emph{Resolução}. Temos por definição:
    \begin{displaymath}
        |x - 1| = \left\{\begin{array}{cc}
             x - 1, \text{ se } x - 1 \ge 0 \implies x \ge 1 &  \\
             -(x-1), \text{ se } x - 1 < 0 \implies x < 1& 
        \end{array} \right.
    \end{displaymath}
    \begin{displaymath}
        |x+2| = \left\{\begin{array}{cc}
             x+2, \text{ se } x + 2 \ge 0 \implies x \ge -2&  \\
             -(x+2), \text{ se } x + 2 < 0 \implies x < -2& 
        \end{array} \right.
    \end{displaymath}
    Apliquemos a partição entre intervalos:
    \begin{itemize}
        \item Se $x < -2$: \emph{Verdadeiro para $x = \dfrac{-1 - \sqrt{21}}{2}$}.
        \begin{align*}
            (-x + 1) \cdot (-x -2) = 3 & \implies x^2 + 2x - x -2 = 3 \\ & \implies
            x^2 + x - 5 = 0
            \\ \text{Fórmula} \\
            x = \frac{-1 \pm \sqrt{1 - (-20)}}{2} & \implies x = \frac{-1 \pm \sqrt{21}}{2}
            \\ \text{Raízes} \\
            x_1 & = \frac{-1 + \sqrt{21}}{2} \\
            x_2 & = \frac{-1 - \sqrt{21}}{2}
        \end{align*}
        \begin{itemize}
            \item Caso $x = \frac{-1 + \sqrt{21}}{2}$: \emph{Falso}. Um número positivo não é menor que um número negativo.
            \begin{align*}
                \frac{-1 + \sqrt{21}}{2} < -2 & \implies -1 + \sqrt{21} < -4 \\ & \implies
                \sqrt{21} < -3
            \end{align*}
            \item Caso $x = \frac{-1 - \sqrt{21}}{2}$: \emph{Verdadeiro.}
            \begin{align*}
                \frac{-1 - \sqrt{21}}{2} < -2 & \implies - \sqrt{21} < -3 
                \\ & \implies
                (-\sqrt{21})^2 < (-3)^2 \\ & \implies 21 > 9
            \end{align*}
        \end{itemize}
        \item Se $-2 \le x < 1$: \emph{Falso}. Não há solução real.
        \begin{align*}
            -(x-1) \cdot (x+2) = 3 & \implies
            (-x + 1) \cdot (x + 2) - 3 = 0 \\ & \implies
            -x^2 -2x +x + 2 - 3 = 0 \\ & \implies
            -x^2 -x - 1 = 0
            \\ \text{Fórmula} \\
            x = \frac{1 \pm \sqrt{1 - 4}}{-2} & = \frac{1 \pm \sqrt{-3}}{-2}
        \end{align*}
        \item Se $x \ge 1$: \emph{Falso.}
        \begin{align*}
            (x - 1) \cdot (x +2) = 3 & \implies x^2 +2x - x + 2 - 3 = 0 \\ & \implies
            x^2 + x - 1 = 0
            \\ \text{Fórmula} \\
            x &= \frac{-1 \pm \sqrt{5}}{2}
            \\ \text{Raízes} \\
            x_1 &= \frac{-1 + \sqrt{5}}{2} \\
            x_2 &= \frac{-1 - \sqrt{5}}{2}
        \end{align*}
        \begin{itemize}
            \item Caso $x = \frac{-1 + \sqrt{5}}{2}$: \emph{Falso.}
            \begin{align*}
                \frac{-1 + \sqrt{5}}{2} \ge 1 & \implies -1 + \sqrt{5} \ge 2
                \\ & \implies \sqrt{5} \ge 3
            \end{align*}
            \item Caso $x = \frac{-1 - \sqrt{5}}{2}$: \emph{Falso.}
            \begin{align*}
                \frac{-1 - \sqrt{5}}{2} \ge 1 & \implies -1 -\sqrt{5} \ge 2
                \\ & \implies 
                -\sqrt{5} \ge 3
            \end{align*}
        \end{itemize}
    \end{itemize}
    Analisando todos os casos podemos concluir, portanto, que o conjunto solução para a equação modular é:
    \begin{displaymath}
        S = \{ \frac{-1 - \sqrt{21}}{2} \}
    \end{displaymath}
    \item $|2x - 5| - 3 \le -2$;
    \\ \emph{Resolução.} Note que:
    \begin{displaymath}
        |2x - 5| -3 \le -2 \implies |2x - 5| \le 1
    \end{displaymath}
    Portanto temos que:
    \begin{align*}
        |2x - 5| \le 1 & \implies -1 \le 2x - 5 \le 1 \\ & \implies
        4 \le 2x \le 6 \\ & \implies
        2 \le x \le 3
    \end{align*}
    Desse modo temos que o conjunto solução desta inequação é:
    \begin{displaymath}
        S = \{ x \in \R; 2 \le x \le 3\}
    \end{displaymath}
    \item $|x^2 - 1| \le 3$;
    \\ \emph{Resolução}. Temos:
    \begin{align*}
        |x^2 - 1| \le 3 & \implies -3 \le x^2 + 1 \le 3 \\ & \implies
        -2 \le x^2 \le 4 \\ & \implies
        0 \le x^2 \le 4 \quad (x^2 \ge 0) \\ & \implies
        \sqrt{x^2} \le \sqrt{4} \\ & \implies
        |x| \le \pm 2 \\ & \implies
        -2 \le x \le 2
    \end{align*}
    Desse modo o conjunto solução para esta inequação é
    \begin{displaymath}
        S = \{x \in \R; -2 \le x \le 2 \}
    \end{displaymath}
    \item $|x + 1| - |x - 1| < -2$;
    \\ \emph{Resolução}.
    Podemos utilizar da partição em intervalos. Note que:
    \begin{align*}
        |x + 1| = \left\{ \begin{array}{cc}
             x + 1, \text{ se } x + 1 \ge 0 \implies x \ge -1&  \\
             -( x + 1), \text{ se } x + 1 < 0 \implies x < -1&
        \end{array}\right.
    \end{align*}
    \begin{align*}
        |x - 1| = \left\{ \begin{array}{cc}
             x - 1, \text{ se } x - 1 \ge 0 \implies x \ge 1&  \\
             -( x - 1), \text{ se } x - 1 < 0 \implies x < 1& 
        \end{array}\right.
    \end{align*}
    Temos os seguintes casos: 
    \begin{itemize}
        \item Se $x < -1$: \emph{Falso}. Não existe solução para este caso.
        \begin{align*}
            |x + 1| - |x - 1| < -2 & \implies
            -x-1 - (-x + 1) < -2 \\ & \implies
            -2 < -2
        \end{align*}
        \item Se $x \ge -1 \text{ e } x < 1$: \emph{Falso}. Chegamos em um absurdo pois $x \ge -1$; não existe solução.
        \begin{align*}
            |x + 1| - |x - 1| < -2 & \implies
            x + 1 - (-x + 1) < -2 \\ & \implies
            2x < -2 \\ & \implies x < -1
        \end{align*}
        \item Se $x \ge 1$: \emph{Falso}. Não existe solução para este caso.
        \begin{align*}
            |x + 1| - |x - 1| < -2 & \implies x + 1 - (x - 1) < -2 \\ & \implies
            2 < -2
        \end{align*}
    \end{itemize}
    Como não encontramos solução para nenhum dos outros casos, temos que o conjunto solução é
    \begin{displaymath}
        S = \emptyset
    \end{displaymath}
\end{enumerate}