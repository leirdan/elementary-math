\enquote{Ache os valores de $x$ para os quais cada uma das seguintes inequações é válida:}
\begin{enumerate}
    \item $x^2 - 9 > 0$;
    \\
    \emph{Resolução}. Temos:
    \begin{align*}
        x^2 - 9 > 0 & \implies (x - 3) \cdot (x + 3) > 0
    \end{align*}
    Fazendo o estudo do sinal, notamos que, se $x < -3$ ou $x > 3$, a desigualdade é válida. Logo, o conjunto solução da inequação é
    \begin{displaymath}
        S = \{ x \in \R; x < -3 \text{ ou } x > 3 \}.
    \end{displaymath}
    \item $\dfrac{x}{x^2 + 9} > 0$;
    \\
    \emph{Resolução}. Temos:
    \begin{align*}
        \frac{x}{x^2 + 9} > 0 & \implies x > 0
        \quad \cdot(x^2 + 9)
    \end{align*}
    Portanto, o conjunto solução que torna a desigualdade verdadeira é
    \begin{displaymath}
        S = \{x; x \in \R_+^* \}
    \end{displaymath}
    \item $\dfrac{x - 3}{x + 1} > 0$;
    \\
    \emph{Resolução}. Fazendo o estudo de sinal de $x - 3, x + 1$ e $\dfrac{x-3}{x+1}$, temos que, para que $x > 0$, o conjunto solução será:
    \begin{displaymath}
        S = \{ x \in \R; x < -1 \text{ ou } x > 3 \}
    \end{displaymath}
    \item $\dfrac{x^2 - 1}{x^2 - 3} > 0$;
    \\
    \emph{Resolução}. Note que $x^2 - 1 = (x - 1) \cdot (x + 1)$ e $x^2 - 3 = (x - \sqrt{3}) \cdot (x + \sqrt{3})$. \\ 
    Fazendo o estudo de sinal de $\dfrac{(x - 1) \cdot (x + 1)}{(x - \sqrt{3}) \cdot (x + \sqrt{3})}$, temos que o conjunto solução para a desigualdade é
    \begin{displaymath}
        S = \{x \in \R; x < -\sqrt{3} \text{ ou } (x > -1 \text{ e } x < 1) \text{ ou } x > \sqrt{3} \}
    \end{displaymath}
    \item $\dfrac{x^2 + x - 6}{x^2 + 6x + 5} \le 0$
    \\
    \emph{Resolução.} Note que $\dfrac{x^2 + x - 6}{x^2 + 6x + 5} = \dfrac{(x - 2) \cdot (x + 3)}{(x + 1) \cdot (x + 5)}$. Fazendo o estudo do sinal dessa fração, temos que o conjunto solução que atende à desigualdade é
    \begin{displaymath}
        S = \{ x \in \R; -5 < x \le -3 \text{ ou } -1 < x \le 2 \}
    \end{displaymath}
    \item $\dfrac{-x^2 - x + 6}{x^2 - 5x + 4} \le 0$
    \\
    \emph{Resolução}. Note que $\dfrac{-x^2 - x + 6}{x^2 - 5x + 4} = \dfrac{(-x -3) \cdot (x - 2)}{(x-4) \cdot (x-1)}$. Fazendo o estudo do sinal, temos que o conjunto solução que atende à desigualdade é
    \begin{displaymath}
        S = \{ x \in \R; x \le -3 \text{ ou } 1 < x \le 2 \text{ ou } x > 4 \}
    \end{displaymath}
\end{enumerate}