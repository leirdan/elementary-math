\enquote{Calcule as dimensões de um triângulo de $16cm$ de perímetro e $15cm^2$ de área.} \\
\emph{Resolução.} Seja $x$ a base do retângulo e $y$ sua altura. Temos a seguinte situação:
\begin{align*}
    2x + 2y = 16 & \\
    x \cdot y = 15 &
\end{align*}
Podemos, então, definir $y = \frac{15}{x}$. Aplicando na primeira equação, temos:
\begin{align*}
    2x + 2\cdot(\frac{15}{x}) = 16 & \implies
    2x + \frac{30}{x} = 16 \\ & \implies
    2x^2 + 30 = 16x \\ & \implies
    2x^2 - 16x + 30 = 0
\end{align*}
Chegamos em uma equação de 2º grau. Vamos resolvê-la:
\begin{align*}
    \text{Discriminante} \\
    \Delta = b^2 - 4ac & \implies \Delta = (-16)^2 - 4\cdot2\cdot30 \\ & \implies
    \Delta = 256 - 240 \\ & \implies \Delta = 16 \\
    \text{Equação}& \\
    x = \frac{-b \pm \sqrt{\Delta}}{2a} & \implies x = \frac{16 \pm \sqrt{16}}{4} \\ & \implies
    x = \frac{16 \pm 4}{4} \\
    \text{Raízes}& \\
    x_1 = \frac{16 + 4}{4} & \implies x_1 = 5 \\
    x_2 = \frac{16 - 4}{4} & \implies x_2 = 3
\end{align*}
Assim, podemos concluir que as raízes $x_1$ e $x_2$ equivalem às dimensões do retângulo, onde $x = x_1 = 5$ e $y = x_2 = 3$.