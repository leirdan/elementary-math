\enquote{Seja $x \in \mathbb{R}$. Mostre que}
\begin{enumerate}
    \item $|x - 5| < 0,1 \implies |2x - 10| < 0,2$;
    \\ \emph{Resolução.} Temos:
    \begin{align*}
        |x - 5| < 0,1 & \implies
        -0,1 < x - 5 < 0,1 \\ & \implies
        -0,2 < 2x - 10 < 0,2 \quad (\cdot 2) \\ & \implies
        |2x - 10| < 0,2
    \end{align*}
    Assim, está provada a implicação.
    \item $|x + 3| < 0,1 \implies |-\frac{3}{2}x + 3 - 7,5| < 0,15$;
    \\ \emph{Resolução.} Temos:
    \begin{align*}
        |x + 3| < 0,1 & \implies -0,1 < x + 3 < 0,1 \\ & \implies
        -0,1 < x + 3 - 7,5 \\ & \implies
        -0,1 < x - 2 + 5 < 0,1 \quad (3 = 5 - 2) \\ & \implies 
        -0,1 \cdot (-\frac{3}{2}) < (x - 2 + 5) \cdot (-\frac{3}{2}) < 0,1 \cdot (-\frac{3}{2}) \\ & \implies
        0,15 > -\frac{3}{2}x + 3 - 7,5 > -0,15 \\ & \implies
        |-\frac{3}{2}x + 3 - 7,5| < 0,15 \quad (\text{"Fecho" o módulo})
    \end{align*}
    Assim, está provada a implicação.
    \item $|x - 2| < \sqrt{5} - 2 \implies |x^2 - 4| < 1$;
    \\ \emph{Não fiz.}
    \item $|x -3| < \sqrt{46} - 5 \implies |x^2 + 4x - 21| < 21$
    \\ \emph{Resolução.} Temos que:
    \begin{align*}
        |x - 3| < \sqrt{46} - 5 & \implies
        -(\sqrt{46} - 5) < x - 3 < \sqrt{46} -5 \\ & \implies
         -\sqrt{46} + 5 < x - 3 < \sqrt{46} -5 \\ & \implies
         -\sqrt{46} +10 < x + 2 < \sqrt{46} \quad (+5) \\ & \implies
         (-\sqrt{46} + 10)^2 < (x + 2)^2 < (\sqrt{46})^2 \\ & \implies
         46 - 20\sqrt{46} + 100 < x^2 + 4x + 4 < 46 \\ & \implies
         21 - 20\sqrt{46} + 100 < x^2 + 4x - 21 < 21 \quad (-21) \\ & \implies
         -21 < 21 - 20 \sqrt{46} + 10 < x^2 + 4x - 21 < 21 \\ & \implies 
         -21 < x^2 + 4x - 21 < 21
         \quad (\text{Transitividade}) \\ & \implies
         |x^2 + 4x - 21| < 21
    \end{align*}
    Assim, está provada a implicação.
\end{enumerate}