\enquote{Considere a matriz 
\begin{displaymath}
    A = \left[\begin{array}{ccc}
         2 & 1 & 0  \\
         1 & -2 & -1 \\
         3 & -5 & -3
    \end{array}\right]
\end{displaymath}
Calcule $A^{-1}$, caso $A$ seja invertível, e calcule as soluções do sistema linear $AX = B$, onde $B = 0_{3\times1}$.}
\\ 
\emph{Resolução}. Calculemos primeiro $\det A$:
\begin{align*}
    \det A &= (12 - 3 + 0) - (0 - 3 - 10) \\ &= 
    9 - (-13) \\ &=
    20
\end{align*}
Como $\det A \ne 0$, então $A$ é invertível e existe $A^{-1}$. Vamos encontrar a inversa:
\begin{align*}
    A^{-1} =& \left[\begin{array}{cccccc}
         2 & 1 & 0 & 1 & 0 & 0 \\
         1 & -2 & -1 & 0 & 1 & 0 \\
         3 & -5 & -3 & 0 & 0 & 1
    \end{array}\right] \quad (l_2 \leftrightarrow l_1) \\ \rightarrow
    &\left[\begin{array}{cccccc}
         1 & -2 & -1 & 0 & 1 & 0 \\
         2 & 1 & 0 & 1 & 0 & 0 \\
         3 & -5 & -3 & 0 & 0 & 1
    \end{array}\right] \quad (l_2 \rightarrow l_2 - 2l_1; l_3 \rightarrow l_3 - 3l_1) \\ \rightarrow
    &\left[\begin{array}{cccccc}
         1 & -2 & -1 & 0 & 1 & 0 \\
         0 & 5 & 2 & 1 & -2 & 0 \\
         0 & 1 & 0 & 0 & -3 & 1
    \end{array}\right] \quad(l_2 \leftrightarrow l_3) \\ \rightarrow
    &\left[\begin{array}{cccccc}
         1 & -2 & -1 & 0 & 1 & 0 \\
         0 & 1 & 0 & 0 & -3 & 1 \\
         0 & 5 & 2 & 1 & -2 & 0
    \end{array}\right] \quad (l_1 \rightarrow l_1 + 2l_2; l_3 \rightarrow l_3 - 5l_2) \\ \rightarrow
    &\left[\begin{array}{cccccc}
         1 & 0 & -1 & 0 & -5 & 2 \\
         0 & 1 & 0 & 0 & -3 & 1 \\
         0 & 0 & 2 & 1 & 13 & -5
    \end{array}\right] \quad (l_3 \rightarrow \frac{l_3}{2}) \\ \rightarrow
    &\left[\begin{array}{cccccc}
         1 & 0 & -1 & 0 & -5 & 2 \\
         0 & 1 & 0 & 0 & -3 & 1 \\
         0 & 0 & 1 & \frac{1}{2} & \frac{13}{2} & -\frac{5}{2}
    \end{array}\right] \quad (l_1 \rightarrow l_1 + l_3) \\ \rightarrow
    &\left[\begin{array}{cccccc}
         1 & 0 & 0 & \frac{1}{2} & \frac{3}{2} & -\frac{1}{2} \\
         0 & 1 & 0 & 0 & -3 & 1 \\
         0 & 0 & 1 & \frac{1}{2} & \frac{13}{2} & -\frac{5}{2}
    \end{array}\right]
\end{align*}
Logo, temos a matriz
\begin{displaymath}
    A^{-1} = \left[\begin{array}{ccc}
         \frac{1}{2} & \frac{3}{2} & -\frac{1}{2} \\
         0 & -3 & 1 \\
         \frac{1}{2} & \frac{13}{2} & -\frac{5}{2}
    \end{array}\right]      
\end{displaymath}
Calculemos, portanto, as soluções do sistema $AX = B$:
\begin{align*}
    AX = B & \implies A^{-1} \cdot A \cdot X = A^{-1} \cdot B \\ & \implies
    X = A^{-1} \cdot B
\end{align*}
Como $B$ é uma matriz $0_{3\times1}$, concluímos que $X$ é também uma matriz $0_{3\times1}$ e o sistema tem solução trivial (0, 0, 0).