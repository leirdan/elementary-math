\enquote{Determine, caso exista, a matriz $A$, tal que $AB = C$, em que}
\begin{displaymath}
B = \left[\begin{array}{ccc}
    1 & -1 \\
    2 & 2 \\
    1 & 0
\end{array} \right]
\text{ e }
C = \left[\begin{array}{cc}
    3 & 1 \\
    -1 & 4 \\
\end{array} \right].
\end{displaymath}
\\ 
\emph{Resolução}. Temos que, para que a multiplicação seja possível, a matriz A é obrigatoriamente do tipo 2x3. Suponhamos a matriz A como
\begin{displaymath}
A = \left[\begin{array}{ccc} 
	x_{11} & x_{12} & x_{13} \\
	x_{21} & x_{22} & x_{23} 
\end{array}\right]
\end{displaymath}
Façamos a multiplicação da matriz A, definida genericamente, pela matriz B:
\begin{align*}
    A \cdot B = C \implies 
    \left[\begin{array}{ccc} 
        x_{11} & x_{12} & x_{13} \\ 
        x_{21} & x_{22} & x_{23} 
    \end{array}\right] \cdot 
    \left[\begin{array}{cc} 
        1 & -1 \\
        2 & 2 \\ 
        1 & 0 
    \end{array}\right] =
    \left[\begin{array}{cc} 
        3 & 1 \\
        -1 & 4 
    \end{array} \right]
\end{align*}
Conhecendo os elementos da matriz C, teremos as seguintes expressões para cada um deles:
\begin{align*}
    (C_{11})& \quad x_{11} + 2\cdot x_{12} + x_{13} = 3 \\
    (C_{12})& \quad -x_{11} + 2\cdot x_{12} = 1  \\
    (C_{21})& \quad x_{21} + 2\cdot x_{22} + x_{23} = -1 \\
    (C_{22})& \quad -x_{21} + 2\cdot x_{22} = 4
\end{align*}
Transformemos essas expressões em dois sistemas diferentes, um para cada linha, e tomemos $k \in \mathbb{N}^*$:
\begin{displaymath}
    C_{1k} = \left\{\begin{array}{ccccc} 
    x_{11} & + 2x_{12} & + x_{13} & = 3  &\\
    -x_{11} &+ 2x_{12} && = 1 &
    \end{array}\right.
\end{displaymath}
\begin{displaymath}
    C_{2k} = \left\{\begin{array}{ccccc} 
    x_{21} & + 2x_{22} & + x_{23} & = -1  &\\
    -x_{21} &+ 2x_{22} && = 4 &
    \end{array}\right.
\end{displaymath}
Analisando $C_{1k}$, teremos a seguinte matriz aumentada:
\begin{align*}
    &\left[\begin{array}{cccc} 
        1 & 2 & 1 & 3 \\
        -1 & 2 & 0 & 1
    \end{array} \right] \quad (l_2 \rightarrow l_2 + l_1) 
    \\ &= 
    \left[\begin{array}{cccc} 
        1 & 2 & 1 & 3 \\
        0 & 4 & 1 & 4
    \end{array} \right] \quad (l_1 \rightarrow l_1 - \frac{1}{2}l_2)
    \\ &=
    \left[\begin{array}{cccc} 
        1 & 0 & \frac{1}{2} & 1 \\
        0 & 4 & 1 & 4
    \end{array} \right] \quad (l_1 \rightarrow l_1 \cdot 2)
    \\ &=
    \left[\begin{array}{cccc} 
        2 & 0 & 1 & 2 \\
        0 & 4 & 1 & 4
    \end{array} \right] 
\end{align*}
Portanto, o sistema deve satisfazer:
\begin{align*}
    (I) \quad 2x_{11} + x_{13} = 2 \implies x_{11} = 1 - \frac{x_{13}}{2} \\
    (II) \quad 4x_{12} + x_{13} = 4 \implies x_{12} = 1 - \frac{x_{13}}{4}
\end{align*}
Assim, a solução para esse sistema, denotada por $S_1$, é:
\begin{displaymath}
    S_1 = (1 - \frac{x_{13}}{2}, 1 -\frac{x_{13}}{4}, x_{13}),
\end{displaymath}
onde $x_{13} \in \mathbb{R}$. \\
Por outro lado, analisando $C_{2k}$, temos a seguinte matriz aumentada:
\begin{align*}
    &\left[\begin{array}{cccc} 
    1 & 2 & 1 & -1 \\
    -1 & 2 & 0 & 4
    \end{array}\right] \quad (l_2 \rightarrow l_2 + l_1)\\ 
    = 
    &\left[\begin{array}{cccc} 
    1 & 2 & 1 & -1 \\
    0 & 4 & 1 & 3
    \end{array}\right] \quad (l_1 \rightarrow 2l_1) \\
    =
    &\left[\begin{array}{cccc} 
    2 & 4 & 2 & -2 \\
    0 & 4 & 1 & 3
    \end{array}\right] \quad (l_1 \rightarrow l_1 - l_2) \\
    =
    &\left[\begin{array}{cccc} 
    2 & 0 & 1 & -5 \\
    0 & 4 & 1 & 3
    \end{array}\right]
\end{align*}
Portanto, o sistema deve satisfazer:
\begin{align*} 
    (I) \quad 2x_{21} + x_{23} = -5 \implies x_{21} = \frac{-5 - x_{23}}{2} \\
    (II) \quad 4x_{22} + x_{23} = 3 \implies x_{22} = \frac{3 - x_{23}}{4}
\end{align*}
Assim, a solução para esse sistema, denotada de $S_2$, será:
\begin{displaymath}
    S_2 = (\frac{-5 - x_{23}}{2}, \frac{3 - x_{23}}{4}, x_{23}),
\end{displaymath}
onde $x_{23} \in \mathbb{R}$. \\
Desse modo, temos que a matriz $A$ existe, tal que:
\begin{align*}
    A = \left[\begin{array}{ccc} 
    1 - \frac{x_{13}}{2} & 1 - \frac{x_{13}}{4} & x_{13} \\
    \frac{-5 - x_{23}}{2} & \frac{3 - x_{23}}{4} & x_{23}
    \end{array} \right]
\end{align*}