\enquote{Sejam $A$ e $B$ matrizes quadradas de ordem $n$. Mostre, através de um contra-exemplo, que a seguinte igualdade não é sempre válida:}
\begin{displaymath}
    \det(A + B) = \det A + \det B.
\end{displaymath}
\emph{Resolução}. Sejam $A, B$ as seguintes matrizes $2\times2$:
\begin{displaymath}
    A = \left[\begin{array}{cc} 
        1 & 2 \\ 
        -1 & 3 
        \end{array}\right], 
    B = \left[\begin{array}{cc} 
        0 & 5 \\
        3 & -1 
        \end{array}\right], 
    A + B = \left[\begin{array}{cc} 
        1 & 7 \\ 
        2 & 2 
        \end{array}\right]
\end{displaymath}
Temos que:
\begin{align*}
    \det A &= (1 \cdot 3) - (2 \cdot -1) = 5 \\
    \det B &= (0 \cdot -1) - (5 \cdot 3) = -15 \\
    \det (A+B) &= (1 \cdot 2) - (7 \cdot 2) = -12
\end{align*}
Desse modo, como $\det A + \det B = -10$, e é diferente de $\det (A + B)$, a igualdade não é sempre válida.