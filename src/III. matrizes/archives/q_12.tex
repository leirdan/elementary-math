\enquote{Considere as matrizes
\begin{displaymath}
    A = \left[
        \begin{array}{cccc}
            3 & 0 & 0 & 0 \\
            -2 & 4 & 0 & 0 \\
            0 & 0 & 1 & 0 \\
            2 & -2 & 2 & -2
        \end{array}
    \right],
    B = \left[
        \begin{array}{cccc}
            1 & -2 & 2 & -1 \\
            0 & 2 & 1 & 3 \\
            0 & 0 & 3 & 1 \\
            0 & 0 & 0 & 1
        \end{array}
    \right]
\end{displaymath}
Calcule:}
\begin{enumerate}
    \item $A \cdot B$ \\
    \emph{Resolução}. Temos a matriz
    \begin{displaymath}
        (A \cdot B) = \left[
            \begin{array}{cccc}
                -3 & -6 & 6 & -3 \\
                -2 & 12 & 0 & 14 \\
                0 & 0 & 3 & 1 \\ 
                -2 & -8 & 8 & -8
            \end{array}
        \right]
    \end{displaymath}
    \item $\det (A \cdot B)$ \\
    \emph{Resolução}. Calculemos o determinante a partir de cofatores:
    \begin{align*}
        \det (A \cdot B) &= \left|
            \begin{array}{cccc}
                -3 & -6 & 6 & -3 \\
                -2 & 12 & 0 & 14 \\
                0 & 0 & 3 & 1 \\ 
                -2 & -8 & 8 & -8
            \end{array}
        \right| \quad (l_2 \rightarrow \frac{l_2}{2}) \\ &=
        2 \cdot \left|
            \begin{array}{cccc}
                -3 & -6 & 6 & -3 \\
                -1 & 6 & 0 & 7 \\
                0 & 0 & 3 & 1 \\ 
                -2 & -8 & 8 & -8
            \end{array} 
        \right| \quad (c_3 \rightarrow c_3 + c_2) \\ &=
        2 \cdot \left|
            \begin{array}{cccc}
                -3 & -6 & 0 & -3 \\
                -1 & 6 & 6 & 7 \\
                0 & 0 & 3 & 1 \\ 
                -2 & -8 & 0 & -8
            \end{array} 
        \right| \quad (c_4 \rightarrow c_4 - \frac{c_3}{3}) \\ &=
        2 \cdot \left|
            \begin{array}{cccc}
                -3 & -6 & 0 & -3 \\
                -1 & 6 & 6 & 5 \\
                0 & 0 & 3 & 0 \\ 
                -2 & -8 & 0 & -8
            \end{array} 
        \right|
    \end{align*}
    Fixando a linha 3, temos:
    \begin{align*}
        \det (A \cdot B) & = 2 \cdot a_{33} \cdot C_{33} \\ & = 2 \cdot 3 \cdot (-1)^{6} \cdot \det \left|\begin{array}{ccc}
            -3 & -6 & -3 \\
            -1 & 6 & 5 \\
            -2 & -8 & -8
        \end{array}\right| \\ & =
        6 \cdot 72 \\ & =
        432
    \end{align*}
    Temos, portanto, que $\det (A \cdot B) = 432$.
    \item $\det (A + B)$ \\
    \emph{Resolução}. Seja 
    \begin{displaymath}
        (A + B) = \left[
            \begin{array}{cccc}
                2 & -2 & 2 & -1 \\
                -2 & 6 & 1 & 3 \\
                0 & 0 & 4 & 1 \\
                2 & -2 & 2 & -1
            \end{array}
        \right]
    \end{displaymath}
    Temos:
    \begin{align*}
        \det (A + B) &= \left|
            \begin{array}{cccc}
                2 & -2 & 2 & -1 \\
                -2 & 6 & 1 & 3 \\
                0 & 0 & 4 & 1 \\
                2 & -2 & 2 & -1
            \end{array}
        \right| \quad (c_1 \rightarrow c_1 + c_2) \\ &=
        \left|
            \begin{array}{cccc}
                0 & -2 & 2 & -1 \\
                4 & 6 & 1 & 3 \\
                0 & 0 & 4 & 1 \\
                0 & -2 & 2 & -1
            \end{array}
        \right| 
    \end{align*}
    Fixando a coluna 1 temos:
    \begin{align*}
        \det (A + B) &= a_{21} \cdot C_{21} \\ 
        &= 4 \cdot (-1) \cdot \det \left|
            \begin{array}{ccc}
                -2 & 2 & -1 \\
                0 & 4 & 1 \\
                -2 & 2 & -1
            \end{array}
        \right| \\
        &= -4 \cdot 0 \\
        &= 0
    \end{align*}
    Portanto, $\det (A + B) = 0$.
    \item A inversa da matriz $(A + B)$, caso exista. \\
    \emph{Resolução}. Como $\det (A + B) = 0$, a matriz $(A + B)$ não é invertível.
    \item A solução do sistema $AX = C$, onde 
    \begin{displaymath}
        C = \left[\begin{array}{c} 3 \\ 2 \\ 1 \\ 0 \end{array}\right].    
    \end{displaymath}
\end{enumerate}